\documentclass{beamer}

\input{es/preamble.tex}

\subtitle{Parte 2: Documentos Estructurados y Más}

\begin{document}

%%%%%%%%%%%%%%%%%%%%%%%%%%%%%%%%%%%%%%%%%%%%%%%%%%%%%%%%%%%%%%%%%%%%%%%%%%%%%%%
%%%%%%%%%%%%%%%%%%%%%%%%%%%%%%%%%%%%%%%%%%%%%%%%%%%%%%%%%%%%%%%%%%%%%%%%%%%%%%%
%%%%%%%%%%%%%%%%%%%%%%%%%%%%%%%%%%%%%%%%%%%%%%%%%%%%%%%%%%%%%%%%%%%%%%%%%%%%%%%
\begin{frame}
\titlepage
\end{frame}

%%%%%%%%%%%%%%%%%%%%%%%%%%%%%%%%%%%%%%%%%%%%%%%%%%%%%%%%%%%%%%%%%%%%%%%%%%%%%%%
%%%%%%%%%%%%%%%%%%%%%%%%%%%%%%%%%%%%%%%%%%%%%%%%%%%%%%%%%%%%%%%%%%%%%%%%%%%%%%%
%%%%%%%%%%%%%%%%%%%%%%%%%%%%%%%%%%%%%%%%%%%%%%%%%%%%%%%%%%%%%%%%%%%%%%%%%%%%%%%
\section{Documentos Estructurados}

%%%%%%%%%%%%%%%%%%%%%%%%%%%%%%%%%%%%%%%%%%%%%%%%%%%%%%%%%%%%%%%%%%%%%%%%%%%%%%%
%%%%%%%%%%%%%%%%%%%%%%%%%%%%%%%%%%%%%%%%%%%%%%%%%%%%%%%%%%%%%%%%%%%%%%%%%%%%%%% 
%%%%%%%%%%%%%%%%%%%%%%%%%%%%%%%%%%%%%%%%%%%%%%%%%%%%%%%%%%%%%%%%%%%%%%%%%%%%%%% 
\begin{frame}{Contenido}
  \begin{multicols}{2}
    \tableofcontents[currentsection]
  \end{multicols}
\end{frame}

%%%%%%%%%%%%%%%%%%%%%%%%%%%%%%%%%%%%%%%%%%%%%%%%%%%%%%%%%%%%%%%%%%%%%%%%%%%%%%%
%%%%%%%%%%%%%%%%%%%%%%%%%%%%%%%%%%%%%%%%%%%%%%%%%%%%%%%%%%%%%%%%%%%%%%%%%%%%%%%
%%%%%%%%%%%%%%%%%%%%%%%%%%%%%%%%%%%%%%%%%%%%%%%%%%%%%%%%%%%%%%%%%%%%%%%%%%%%%%%
\begin{frame}{\insertsection}
  \begin{itemize}
  \item En la Parte 1, aprendimos acerca de los comandos y entornos
    para la tipografía de texto y contenido matemático.
  \item Ahora, vamos a conocer acerca de los comandos y entornos para
    generar documentos estructurados.
  \item Puede probar los nuevos comandos en Overleaf:
  \end{itemize}
  \vskip 2em
  \begin{center}
    \fbox{\href{\wlnewdoc{basics.tex}}{%
        Click aquí para abrir el documento de ejemplo en \wllogo{}}}
    \\[1ex]\scriptsize{}
    Para un mejor resultado, use
      \href{http://www.google.com/chrome}{Google Chrome} o una versión
      actualizada de \href{http://www.mozilla.org/en-US/firefox/new/}{FireFox}.
  \end{center}
  \vskip 2ex
  \begin{itemize}
  \item Vamos a comenzar!
  \end{itemize}
\end{frame}

%%%%%%%%%%%%%%%%%%%%%%%%%%%%%%%%%%%%%%%%%%%%%%%%%%%%%%%%%%%%%%%%%%%%%%%%%%%%%%%
%%%%%%%%%%%%%%%%%%%%%%%%%%%%%%%%%%%%%%%%%%%%%%%%%%%%%%%%%%%%%%%%%%%%%%%%%%%%%%%
%%%%%%%%%%%%%%%%%%%%%%%%%%%%%%%%%%%%%%%%%%%%%%%%%%%%%%%%%%%%%%%%%%%%%%%%%%%%%%% 
\subsection{Título y Resumen}
\begin{frame}[fragile]{\insertsubsection}
  \begin{itemize}{\small
    \item Le decimos a \LaTeX{} el \cmdbs{title} y nombre del
      \cmdbs{author} en el preámbulo.
    \item Luego utilizamos el comando \cmdbs{maketitle} en el
      documento para  visualizarlos en la salida.
    \item Utilice el entorno \bftt{abstract} para crear un resumen.
    }\end{itemize}
  \begin{minipage}{0.55\linewidth}
    \inputminted[fontsize=\scriptsize,frame=single,resetmargins]{latex}%
    {es/structure-title.tex}
  \end{minipage}
  \begin{minipage}{0.35\linewidth}
    \includegraphics[width=\textwidth,clip,trim=2.2in 7in 2.2in 2in]{structure-title.pdf}
  \end{minipage}
\end{frame}

%%%%%%%%%%%%%%%%%%%%%%%%%%%%%%%%%%%%%%%%%%%%%%%%%%%%%%%%%%%%%%%%%%%%%%%%%%%%%%% 
%%%%%%%%%%%%%%%%%%%%%%%%%%%%%%%%%%%%%%%%%%%%%%%%%%%%%%%%%%%%%%%%%%%%%%%%%%%%%%%
%%%%%%%%%%%%%%%%%%%%%%%%%%%%%%%%%%%%%%%%%%%%%%%%%%%%%%%%%%%%%%%%%%%%%%%%%%%%%%%
\subsection{Secciones}
\begin{frame}{\insertsubsection}
  \begin{itemize}{\small
    \item Solo utilice \cmdbs{section} y \cmdbs{subsection}.
    \item ¿Pueden adivinar qué hacen los comandos \cmdbs{section*} y \cmdbs{subsection*}?
    }\end{itemize}
  \begin{minipage}{0.55\linewidth}
    \inputminted[fontsize=\scriptsize,frame=single,resetmargins]{latex}%
    {es/structure-sections.tex}
  \end{minipage}
  \begin{minipage}{0.35\linewidth}
    \includegraphics[width=\textwidth,clip,trim=1.5in 6in 4in 1in]{structure-sections.pdf}
  \end{minipage}
\end{frame}

%%%%%%%%%%%%%%%%%%%%%%%%%%%%%%%%%%%%%%%%%%%%%%%%%%%%%%%%%%%%%%%%%%%%%%%%%%%%%%%
%%%%%%%%%%%%%%%%%%%%%%%%%%%%%%%%%%%%%%%%%%%%%%%%%%%%%%%%%%%%%%%%%%%%%%%%%%%%%%%
%%%%%%%%%%%%%%%%%%%%%%%%%%%%%%%%%%%%%%%%%%%%%%%%%%%%%%%%%%%%%%%%%%%%%%%%%%%%%%%
\subsection{Etiquetas y Referencias Cruzadas}
\begin{frame}[fragile]{\insertsubsection}
  \begin{itemize}{\small
    \item Utilice \cmdbs{label} y \cmdbs{ref} para la numeración automática.
    \item El paquete \bftt{amsmath} proporciona \cmdbs{eqref} para
      las referencias de ecuaciones.
    }\end{itemize}
  \begin{minipage}{0.55\linewidth}
    \inputminted[fontsize=\scriptsize,frame=single,resetmargins]{latex}%
    {es/structure-crossref.tex}
  \end{minipage}
  \begin{minipage}{0.35\linewidth}
    \includegraphics[width=\textwidth,clip,trim=1.8in 6in 1.6in 1in]{structure-crossref.pdf}
  \end{minipage}
\end{frame}

%%%%%%%%%%%%%%%%%%%%%%%%%%%%%%%%%%%%%%%%%%%%%%%%%%%%%%%%%%%%%%%%%%%%%%%%%%%%%%%
%%%%%%%%%%%%%%%%%%%%%%%%%%%%%%%%%%%%%%%%%%%%%%%%%%%%%%%%%%%%%%%%%%%%%%%%%%%%%%%
%%%%%%%%%%%%%%%%%%%%%%%%%%%%%%%%%%%%%%%%%%%%%%%%%%%%%%%%%%%%%%%%%%%%%%%%%%%%%%%
\subsection{Ejercicio}
\begin{frame}[fragile]{Ejercicio de Documentos Estructurados}
  
  \begin{block}{Escriba este pequeño artículo en \LaTeX:
      \footnote{Desde \url{http://pdos.csail.mit.edu/scigen/}, un
        generador aleatorio de artículos.}}
    \begin{center}
      \fbox{\href{\fileuri/structure-exercise-solution.pdf}{%
          Click para abrir el artículo}}
    \end{center}
    Haga su versión del artículo mirando el documento original. Utilice
     \cmdbs{ref} y \cmdbs{eqref} para evitar escribir explícitamente
     la sección y el número de ecuación dentro del texto.
  \end{block}
  \vskip 2ex
  \begin{center}
    \fbox{\href{\wlnewdoc{es/structure-exercise.tex}}{%
        Click para abrir el ejercicio en  \wllogo{}}}
  \end{center}
  
  \begin{itemize}
  \item Una vez que lo haya probado,
    \fbox{\href{\wlnewdoc{es/structure-exercise-solution.tex}}{%
        click aquí para ver la solución}}.
  \end{itemize}
\end{frame}

%%%%%%%%%%%%%%%%%%%%%%%%%%%%%%%%%%%%%%%%%%%%%%%%%%%%%%%%%%%%%%%%%%%%%%%%%%%%%%%
%%%%%%%%%%%%%%%%%%%%%%%%%%%%%%%%%%%%%%%%%%%%%%%%%%%%%%%%%%%%%%%%%%%%%%%%%%%%%%%
%%%%%%%%%%%%%%%%%%%%%%%%%%%%%%%%%%%%%%%%%%%%%%%%%%%%%%%%%%%%%%%%%%%%%%%%%%%%%%%
\section{Figuras y Tablas}

%%%%%%%%%%%%%%%%%%%%%%%%%%%%%%%%%%%%%%%%%%%%%%%%%%%%%%%%%%%%%%%%%%%%%%%%%%%%%%%
%%%%%%%%%%%%%%%%%%%%%%%%%%%%%%%%%%%%%%%%%%%%%%%%%%%%%%%%%%%%%%%%%%%%%%%%%%%%%%%
%%%%%%%%%%%%%%%%%%%%%%%%%%%%%%%%%%%%%%%%%%%%%%%%%%%%%%%%%%%%%%%%%%%%%%%%%%%%%%%
\begin{frame}{Contenido}
  \begin{multicols}{2}
    \tableofcontents[currentsection]
  \end{multicols}
\end{frame}

%%%%%%%%%%%%%%%%%%%%%%%%%%%%%%%%%%%%%%%%%%%%%%%%%%%%%%%%%%%%%%%%%%%%%%%%%%%%%%%
%%%%%%%%%%%%%%%%%%%%%%%%%%%%%%%%%%%%%%%%%%%%%%%%%%%%%%%%%%%%%%%%%%%%%%%%%%%%%%%
%%%%%%%%%%%%%%%%%%%%%%%%%%%%%%%%%%%%%%%%%%%%%%%%%%%%%%%%%%%%%%%%%%%%%%%%%%%%%%%
\subsection{Gráficos}
\begin{frame}[fragile]{\insertsubsection}
  \begin{itemize}
  \item Requiere del paquete \bftt{graphicx}, que proporciona el
    comando \cmdbs{includegraphics}.
  \item Los formatos gráficos soportados incluyen JPEG, PNG y PDF.
  \end{itemize}
  \begin{exampletwouptiny}
\includegraphics[
width=0.5\textwidth]{es/big_chick}

\includegraphics[
width=0.3\textwidth,
angle=270]{es/big_chick}
  \end{exampletwouptiny}
  
  \tiny{Imagen desde \url{http://www.andy-roberts.net/writing/latex/importing_images}}
\end{frame}

%%%%%%%%%%%%%%%%%%%%%%%%%%%%%%%%%%%%%%%%%%%%%%%%%%%%%%%%%%%%%%%%%%%%%%%%%%%%%%% 
%%%%%%%%%%%%%%%%%%%%%%%%%%%%%%%%%%%%%%%%%%%%%%%%%%%%%%%%%%%%%%%%%%%%%%%%%%%%%%% 
%%%%%%%%%%%%%%%%%%%%%%%%%%%%%%%%%%%%%%%%%%%%%%%%%%%%%%%%%%%%%%%%%%%%%%%%%%%%%%%
\begin{frame}[fragile]{Intermedio: Argumentos Opcionales}
  \begin{itemize}
  \item Utilizamos corchetes  \keystrokebftt{[} \keystrokebftt{]} para
    los argumentos opcionales, en lugar de las llaves \keystrokebftt{\{} \keystrokebftt{\}}.
  \item \cmdbs{includegraphics} acepta argumentos opcionales que
    permiten transformar la imagen cuando se incluya. Por ejemplo,
    \bftt{width=0.3\cmdbs{textwidth}}  hace que la imagen ocupe el
    30\% del ancho total asignado para el texto (\cmdbs{textwidth}).
  \item \cmdbs{documentclass} también acepta argumentos
    opcionales. Por ejemplo:
    \mint{latex}|\documentclass[12pt,twocolumn]{article}|
    \vskip 3ex
    hace al texto más grande (12pt) y lo coloca en dos columnas.
  \item ¿Dónde encontramos información sobre estas cosas? Vea las
    diapositivas hasta el final para obtener enlaces a más información.
  \end{itemize}
\end{frame}

%%%%%%%%%%%%%%%%%%%%%%%%%%%%%%%%%%%%%%%%%%%%%%%%%%%%%%%%%%%%%%%%%%%%%%%%%%%%%%%
%%%%%%%%%%%%%%%%%%%%%%%%%%%%%%%%%%%%%%%%%%%%%%%%%%%%%%%%%%%%%%%%%%%%%%%%%%%%%%%
%%%%%%%%%%%%%%%%%%%%%%%%%%%%%%%%%%%%%%%%%%%%%%%%%%%%%%%%%%%%%%%%%%%%%%%%%%%%%%%
\subsection[fragile]{Flotantes}
\begin{frame}{\insertsubsection}
  \begin{itemize}
  \item Permita que \LaTeX{} decida dónde ubicar las figuras.
  \item Puede también darle a la figura un título, una etiqueta y así
    ser referenciado con \cmdbs{ref}.
  \end{itemize}
  \begin{minipage}{0.55\linewidth}
    \inputminted[fontsize=\scriptsize,frame=single,resetmargins]{latex}%
    {es/media-graphics.tex}
  \end{minipage}
  \begin{minipage}{0.35\linewidth}
    \includegraphics[width=\textwidth,clip,trim=2in 5in 3in 1in]{media-graphics.pdf}
  \end{minipage}
\end{frame}

%%%%%%%%%%%%%%%%%%%%%%%%%%%%%%%%%%%%%%%%%%%%%%%%%%%%%%%%%%%%%%%%%%%%%%%%%%%%%%%
%%%%%%%%%%%%%%%%%%%%%%%%%%%%%%%%%%%%%%%%%%%%%%%%%%%%%%%%%%%%%%%%%%%%%%%%%%%%%%%
%%%%%%%%%%%%%%%%%%%%%%%%%%%%%%%%%%%%%%%%%%%%%%%%%%%%%%%%%%%%%%%%%%%%%%%%%%%%%%%
\subsection{Tablas}
\begin{frame}[fragile]{\insertsubsection}
  \begin{itemize}
  \item Las tablas en \LaTeX{} requieren un tiempo para acostumbrarse.
%  \item Utilice el entorno \bftt{tabular} desde el paquete
%\bftt{tabularx}.
  \item El argumento especifica la alineación de las columnas ---
\textbf{l}eft, \textbf{r}ight, \textbf{r}ight.
    \begin{exampletwouptiny}
\begin{tabular}{lrr}
  Art.   & Cant. & Uni. \$ \\
  DVD    & 1     & 19.99   \\
  Sonido & 2     & 39.99   \\
  Cable  & 3     & 1.99    \\
\end{tabular}
    \end{exampletwouptiny}
  \item También se especifican las líneas verticales; utilice el
comando \cmdbs{hline} para las líneas horizontales.
    \begin{exampletwouptiny}
\begin{tabular}{|l|r|r|}    \hline
  Art.   & Cant. & Uni.\$ \\\hline
  DVD    & 1     & 19.99  \\
  Sonido & 2     & 39.99  \\
  Cable  & 3     & 1.99   \\\hline
\end{tabular}
    \end{exampletwouptiny}
  \item Utilice un ampersand \keystrokebftt{\&} para separarlas
columnas y una doble barra invertida
\keystrokebftt{\bs}\keystrokebftt{\bs} para comenzar una nueva
fila(como en el entorno \bftt{align*} visto en la Parte 1).
  \end{itemize}
\end{frame}

%%%%%%%%%%%%%%%%%%%%%%%%%%%%%%%%%%%%%%%%%%%%%%%%%%%%%%%%%%%%%%%%%%%%%%%%%%%%%%% 
%%%%%%%%%%%%%%%%%%%%%%%%%%%%%%%%%%%%%%%%%%%%%%%%%%%%%%%%%%%%%%%%%%%%%%%%%%%%%%% 
%%%%%%%%%%%%%%%%%%%%%%%%%%%%%%%%%%%%%%%%%%%%%%%%%%%%%%%%%%%%%%%%%%%%%%%%%%%%%%%
\addtocontents{toc}{\newpage}
\section{Bibliografías}

%%%%%%%%%%%%%%%%%%%%%%%%%%%%%%%%%%%%%%%%%%%%%%%%%%%%%%%%%%%%%%%%%%%%%%%%%%%%%%%
%%%%%%%%%%%%%%%%%%%%%%%%%%%%%%%%%%%%%%%%%%%%%%%%%%%%%%%%%%%%%%%%%%%%%%%%%%%%%%%
%%%%%%%%%%%%%%%%%%%%%%%%%%%%%%%%%%%%%%%%%%%%%%%%%%%%%%%%%%%%%%%%%%%%%%%%%%%%%%%
\begin{frame}{Contenido}
  \begin{multicols}{2}
    \tableofcontents[currentsection]
  \end{multicols}
\end{frame}

%%%%%%%%%%%%%%%%%%%%%%%%%%%%%%%%%%%%%%%%%%%%%%%%%%%%%%%%%%%%%%%%%%%%%%%%%%%%%%%
%%%%%%%%%%%%%%%%%%%%%%%%%%%%%%%%%%%%%%%%%%%%%%%%%%%%%%%%%%%%%%%%%%%%%%%%%%%%%%%
%%%%%%%%%%%%%%%%%%%%%%%%%%%%%%%%%%%%%%%%%%%%%%%%%%%%%%%%%%%%%%%%%%%%%%%%%%%%%%%
\subsection{bib\TeX}
\begin{frame}[fragile]{\insertsubsection{} 1}
  \begin{itemize}
  \item Colocar las referencias en un archivo \bftt{.bib} en el
    formato de base de datos  `bibtex':
    \inputminted[fontsize=\scriptsize,frame=single]{latex}{es/bib-example.bib}
  \item La mayoría de los gestores de referencias pueden exportar al
    formato bibtex.
  \end{itemize}
\end{frame}

%%%%%%%%%%%%%%%%%%%%%%%%%%%%%%%%%%%%%%%%%%%%%%%%%%%%%%%%%%%%%%%%%%%%%%%%%%%%%%% 
%%%%%%%%%%%%%%%%%%%%%%%%%%%%%%%%%%%%%%%%%%%%%%%%%%%%%%%%%%%%%%%%%%%%%%%%%%%%%%% 
%%%%%%%%%%%%%%%%%%%%%%%%%%%%%%%%%%%%%%%%%%%%%%%%%%%%%%%%%%%%%%%%%%%%%%%%%%%%%%% 
\begin{frame}[fragile]{\insertsubsection{} 2}
  \begin{itemize}
  \item Cada entrada en el archivo  \bftt{.bib} tiene una \emph{clave}
    que puede usar para ser citado en el documento. Por ejemplo,
    \bftt{Jacobson1999Towards} es la clave para este artículo:
    \begin{minted}[fontsize=\small,frame=single]{latex}
@Article{Jacobson1999Towards,
  author = {Van Jacobson},
  ...
}
    \end{minted}
  \item Es recomendable utilizar una clave basada en el nombre, año y
    título del artículo.
  \item \LaTeX{} puede formatear automáticamente sus citas en el texto
    y generar una lista de referencias; basados en estilos estándares,
    y hasta se pueden diseñar sus propios estilos.
  \end{itemize}
\end{frame}

%%%%%%%%%%%%%%%%%%%%%%%%%%%%%%%%%%%%%%%%%%%%%%%%%%%%%%%%%%%%%%%%%%%%%%%%%%%%%%% 
%%%%%%%%%%%%%%%%%%%%%%%%%%%%%%%%%%%%%%%%%%%%%%%%%%%%%%%%%%%%%%%%%%%%%%%%%%%%%%%
%%%%%%%%%%%%%%%%%%%%%%%%%%%%%%%%%%%%%%%%%%%%%%%%%%%%%%%%%%%%%%%%%%%%%%%%%%%%%%%
\begin{frame}[fragile]{\insertsubsection{} 3}
  \begin{itemize}
  \item Utilice el paquete \bftt{natbib} con \cmdbs{citet} y \cmdbs{citep}.
  \item Las referencias bibliográficas van al final del texto con el
    comando \cmdbs{bibliography}, y luego se especifica el estilo con
    \cmdbs{bibliographystyle}.
  \end{itemize}
  \begin{minipage}{0.55\linewidth}
    \inputminted[fontsize=\scriptsize,frame=single,resetmargins]{latex}%
    {es/bib-example.tex}
  \end{minipage}
  \begin{minipage}{0.35\linewidth}
    \includegraphics[width=\textwidth,clip,trim=1.8in 5in 1.8in 1in]{bib-example.pdf}
  \end{minipage}
\end{frame}

%%%%%%%%%%%%%%%%%%%%%%%%%%%%%%%%%%%%%%%%%%%%%%%%%%%%%%%%%%%%%%%%%%%%%%%%%%%%%%% 
%%%%%%%%%%%%%%%%%%%%%%%%%%%%%%%%%%%%%%%%%%%%%%%%%%%%%%%%%%%%%%%%%%%%%%%%%%%%%%%
%%%%%%%%%%%%%%%%%%%%%%%%%%%%%%%%%%%%%%%%%%%%%%%%%%%%%%%%%%%%%%%%%%%%%%%%%%%%%%%
\subsection{Ejercicio}
\begin{frame}[fragile]{Ejercicio: Coloque Todo Junto}
  
  Agregue una imagen y una bibliografía al artículo desde el ejercicio previo.
  
  \begin{enumerate}
  \item Descargue estos archivos de ejemplos a su computadora.

    \begin{center}
      \fbox{\href{\fileuri/big_chick.png?dl=1}{Click aquí para
          descargar imagen}}
      
      \fbox{\href{\fileuri/bib-exercise.bib?dl=1}{Click aquí para
          descargar el archivo bib}}
    \end{center}
    
  \item Súbalos a  Overleaf (Utilice el menú ``project'').
    
  \end{enumerate}
\end{frame}

%%%%%%%%%%%%%%%%%%%%%%%%%%%%%%%%%%%%%%%%%%%%%%%%%%%%%%%%%%%%%%%%%%%%%%%%%%%%%%% 
%%%%%%%%%%%%%%%%%%%%%%%%%%%%%%%%%%%%%%%%%%%%%%%%%%%%%%%%%%%%%%%%%%%%%%%%%%%%%%%
%%%%%%%%%%%%%%%%%%%%%%%%%%%%%%%%%%%%%%%%%%%%%%%%%%%%%%%%%%%%%%%%%%%%%%%%%%%%%%% 
\section{¿Qué Sigue?}

%%%%%%%%%%%%%%%%%%%%%%%%%%%%%%%%%%%%%%%%%%%%%%%%%%%%%%%%%%%%%%%%%%%%%%%%%%%%%%%
%%%%%%%%%%%%%%%%%%%%%%%%%%%%%%%%%%%%%%%%%%%%%%%%%%%%%%%%%%%%%%%%%%%%%%%%%%%%%%%
%%%%%%%%%%%%%%%%%%%%%%%%%%%%%%%%%%%%%%%%%%%%%%%%%%%%%%%%%%%%%%%%%%%%%%%%%%%%%%%
\begin{frame}{Contenido}
  \begin{multicols}{2}
    \tableofcontents[currentsection]
  \end{multicols}
\end{frame}

%%%%%%%%%%%%%%%%%%%%%%%%%%%%%%%%%%%%%%%%%%%%%%%%%%%%%%%%%%%%%%%%%%%%%%%%%%%%%%% 
%%%%%%%%%%%%%%%%%%%%%%%%%%%%%%%%%%%%%%%%%%%%%%%%%%%%%%%%%%%%%%%%%%%%%%%%%%%%%%%
%%%%%%%%%%%%%%%%%%%%%%%%%%%%%%%%%%%%%%%%%%%%%%%%%%%%%%%%%%%%%%%%%%%%%%%%%%%%%%%
\subsection{Cosas Más Esmeradas}
\begin{frame}[fragile]{\insertsubsection}
  \begin{itemize}
  \item Agregue el comando  \cmdbs{tableofcontents} para generar una
    tabla de contenidos.
    
  \item Cambie la clase de documento (\cmdbs{documentclass}) a 
    \mint{latex}!\documentclass{scrartcl}! o 
    \mint{latex}!\documentclass[12pt]{IEEEtran}!
    
  \item Defina su propio comando para una ecuación compleja:
    \begin{exampletwouptiny}
      \newcommand{\rperf}{%
  \rho_{\text{perf}}}
$$
\rperf = {\bf c}'{\bf X} + \varepsilon
$$
    \end{exampletwouptiny}
  \end{itemize}
\end{frame}

%%%%%%%%%%%%%%%%%%%%%%%%%%%%%%%%%%%%%%%%%%%%%%%%%%%%%%%%%%%%%%%%%%%%%%%%%%%%%%% 
%%%%%%%%%%%%%%%%%%%%%%%%%%%%%%%%%%%%%%%%%%%%%%%%%%%%%%%%%%%%%%%%%%%%%%%%%%%%%%% 
%%%%%%%%%%%%%%%%%%%%%%%%%%%%%%%%%%%%%%%%%%%%%%%%%%%%%%%%%%%%%%%%%%%%%%%%%%%%%%% 
\subsection{Otros Paquetes Interesantes}
\begin{frame}{\insertsubsection}
  \begin{itemize}
  \item \bftt{beamer}: para presentaciones
  \item \bftt{todonotes}: comentarios y manejo de ``TODO''(para hacer)
  \item \bftt{listings}: impresora de código fuente para \LaTeX
  \end{itemize}
  Ver \url{https://www.overleaf.com/latex/examples} y \url{http://texample.net}
  para obtener ejemplos de la mayoría de estos paquetes.
\end{frame}

%%%%%%%%%%%%%%%%%%%%%%%%%%%%%%%%%%%%%%%%%%%%%%%%%%%%%%%%%%%%%%%%%%%%%%%%%%%%%%% 
%%%%%%%%%%%%%%%%%%%%%%%%%%%%%%%%%%%%%%%%%%%%%%%%%%%%%%%%%%%%%%%%%%%%%%%%%%%%%%% 
%%%%%%%%%%%%%%%%%%%%%%%%%%%%%%%%%%%%%%%%%%%%%%%%%%%%%%%%%%%%%%%%%%%%%%%%%%%%%%% 
\subsection{Instalación de  \LaTeX{}}
\begin{frame}{\insertsubsection}
  \begin{itemize}
  \item Para ejecutar \LaTeX{} sobre su computadora, deberá contar con
    una \emph{distribución} de las que se encuentran disponible para
    diferentes plataformas. Una distribución incluye un programa
    \bftt{latex} y (típicamente) varios miles de paquetes.
    \begin{itemize}
    \item sobre Windows: \href{http://miktex.org/}{Mik\TeX} o \href{http://tug.org/texlive/}{\TeX Live}
    \item Sobre GNU/Linux: \href{http://tug.org/texlive/}{\TeX Live}
    \item Sobre Mac: \href{http://tug.org/mactex/}{Mac\TeX}
    \end{itemize}
  \item También querrá un editor de texto con sporte para
\LaTeX{}. Vea
\url{http://en.wikipedia.org/wiki/Comparison_of_TeX_editors} para una
lista de muchas opciones.
  \item También tiene que saber más acerca de cómo  \bftt{latex}, y sus
    herramientas relacionadas, trabajan --- consulte las fuentes de la
    siguiente diapositiva. 
  \end{itemize}
\end{frame}

%%%%%%%%%%%%%%%%%%%%%%%%%%%%%%%%%%%%%%%%%%%%%%%%%%%%%%%%%%%%%%%%%%%%%%%%%%%%%%% 
%%%%%%%%%%%%%%%%%%%%%%%%%%%%%%%%%%%%%%%%%%%%%%%%%%%%%%%%%%%%%%%%%%%%%%%%%%%%%%%
%%%%%%%%%%%%%%%%%%%%%%%%%%%%%%%%%%%%%%%%%%%%%%%%%%%%%%%%%%%%%%%%%%%%%%%%%%%%%%%
\subsection{Recursos en Línea}
\begin{frame}{\insertsubsection}
  \begin{itemize}
  \item \href{http://en.wikibooks.org/wiki/LaTeX}{La WikiBook de
      \LaTeX{}} --- excelente tutoriales y materiales de referencia.
  \item \href{http://tex.stackexchange.com/}{\TeX{} Stack Exchange} ---
    haga sus consultas y obtenga excelentes respuestas con una rapidez
    increíble.
  \item \href{http://www.latex-community.org/}{Comunidad \LaTeX{}} ---
    un gran foro en línea
  \item \href{http://ctan.org/}{Comprehensive \TeX{} Archive Network (CTAN)} ---
    más de cuatro mil paquetes, y sus respectivas documentaciones.
  \item Sí utiliza Google seguramente llegará  a uno de los anteriores sitios.
  \end{itemize}
\end{frame}

%%%%%%%%%%%%%%%%%%%%%%%%%%%%%%%%%%%%%%%%%%%%%%%%%%%%%%%%%%%%%%%%%%%%%%%%%%%%%%% 
%%%%%%%%%%%%%%%%%%%%%%%%%%%%%%%%%%%%%%%%%%%%%%%%%%%%%%%%%%%%%%%%%%%%%%%%%%%%%%%
%%%%%%%%%%%%%%%%%%%%%%%%%%%%%%%%%%%%%%%%%%%%%%%%%%%%%%%%%%%%%%%%%%%%%%%%%%%%%%%


\end{document}
