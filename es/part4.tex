\documentclass[draft]{beamer}

\input{preamble2.tex} %este preábulo es agregado por guanucoluis

\subtitle{Parte 4: Adaptando el documento a nuestras necesidades}

\begin{document}

%%%%%%%%%%%%%%%%%%%%%%%%%%%%%%%%%%%%%%%%%%%%%%%%%%%%%%%%%%%%%%%%%%%%%%%%%%%%%%%
%%%%%%%%%%%%%%%%%%%%%%%%%%%%%%%%%%%%%%%%%%%%%%%%%%%%%%%%%%%%%%%%%%%%%%%%%%%%%%%
%%%%%%%%%%%%%%%%%%%%%%%%%%%%%%%%%%%%%%%%%%%%%%%%%%%%%%%%%%%%%%%%%%%%%%%%%%%%%%%
\begin{frame}
  \titlepage
\end{frame}

%%%%%%%%%%%%%%%%%%%%%%%%%%%%%%%%%%%%%%%%%%%%%%%%%%%%%%%%%%%%%%%%%%%%%%%%%%%%%%%
%%%%%%%%%%%%%%%%%%%%%%%%%%%%%%%%%%%%%%%%%%%%%%%%%%%%%%%%%%%%%%%%%%%%%%%%%%%%%%%
%%%%%%%%%%%%%%%%%%%%%%%%%%%%%%%%%%%%%%%%%%%%%%%%%%%%%%%%%%%%%%%%%%%%%%%%%%%%%%%
\section{Estilo De Las Páginas}

%%%%%%%%%%%%%%%%%%%%%%%%%%%%%%%%%%%%%%%%%%%%%%%%%%%%%%%%%%%%%%%%%%%%%%%%%%%%%%%
%%%%%%%%%%%%%%%%%%%%%%%%%%%%%%%%%%%%%%%%%%%%%%%%%%%%%%%%%%%%%%%%%%%%%%%%%%%%%%%
%%%%%%%%%%%%%%%%%%%%%%%%%%%%%%%%%%%%%%%%%%%%%%%%%%%%%%%%%%%%%%%%%%%%%%%%%%%%%%%
\begin{frame}[fragile]{\insertsection: Nativos}
  \begin{itemize}
  \item \LaTeX{} soporta diferentes combinaciones de cabeceras y
    pies de páginas. \cmdbs{pagestyle} define cuál emplearse.
    \begin{itemize}
    \item \bftt{empty}
    \item \bftt{plain}
    \item \bftt{headings}      
    \item \bftt{myheadings}      
    \end{itemize}
  \item Es posible cambiar el estilo de la página actual con la orden
    pies de páginas \cmdbs{thispagestyle}.
  \end{itemize}
\end{frame}

%%%%%%%%%%%%%%%%%%%%%%%%%%%%%%%%%%%%%%%%%%%%%%%%%%%%%%%%%%%%%%%%%%%%%%%%%%%%%%%
%%%%%%%%%%%%%%%%%%%%%%%%%%%%%%%%%%%%%%%%%%%%%%%%%%%%%%%%%%%%%%%%%%%%%%%%%%%%%%%
%%%%%%%%%%%%%%%%%%%%%%%%%%%%%%%%%%%%%%%%%%%%%%%%%%%%%%%%%%%%%%%%%%%%%%%%%%%%%%% 
\begin{frame}[fragile]{\insertsection: Personalizados}
  El estilo \bftt{myheadings} permite modificar el contenido de
  la cabecera.\\[2ex]
  \begin{minipage}{0.5\linewidth}
    \inputminted[fontsize=\scriptsize,frame=single,resetmargins]{latex}%
    {pagestyle-example-myheadings.tex}
  \end{minipage}
  \begin{minipage}{0.4\linewidth}
    % trim: l b r t
    \includegraphics[width=\textwidth,clip,trim=1in 1.1in 1in 1.1in,page=6]{pagestyle-example-myheadings.pdf}
  \end{minipage}
\end{frame}

%%%%%%%%%%%%%%%%%%%%%%%%%%%%%%%%%%%%%%%%%%%%%%%%%%%%%%%%%%%%%%%%%%%%%%%%%%%%%%% 
%%%%%%%%%%%%%%%%%%%%%%%%%%%%%%%%%%%%%%%%%%%%%%%%%%%%%%%%%%%%%%%%%%%%%%%%%%%%%%%
%%%%%%%%%%%%%%%%%%%%%%%%%%%%%%%%%%%%%%%%%%%%%%%%%%%%%%%%%%%%%%%%%%%%%%%%%%%%%%%

\begin{frame}[fragile]{\insertsection: Personalizados}
  El paquete \bftt{fancyhdr} provee comandos para definir el
  contenido del lado izquierdo, centro y derecho, tanto del
  encabezado como el pie de página.\\[2ex]
  \begin{minipage}{0.5\linewidth}
    \inputminted[fontsize=\tiny,frame=single,resetmargins]{latex}%
    {pagestyle-example-fancyhdr.tex}
  \end{minipage}
  \begin{minipage}{0.4\linewidth}
    % trim: l b r t
    \includegraphics[width=\textwidth,clip,trim=1in 1.1in 1in 1.1in,page=6]{pagestyle-example-fancyhdr.pdf}
  \end{minipage}
\end{frame}

%%%%%%%%%%%%%%%%%%%%%%%%%%%%%%%%%%%%%%%%%%%%%%%%%%%%%%%%%%%%%%%%%%%%%%%%%%%%%%%
%%%%%%%%%%%%%%%%%%%%%%%%%%%%%%%%%%%%%%%%%%%%%%%%%%%%%%%%%%%%%%%%%%%%%%%%%%%%%%%
%%%%%%%%%%%%%%%%%%%%%%%%%%%%%%%%%%%%%%%%%%%%%%%%%%%%%%%%%%%%%%%%%%%%%%%%%%%%%%%

\section{Código Fuente en \LaTeX}

%%%%%%%%%%%%%%%%%%%%%%%%%%%%%%%%%%%%%%%%%%%%%%%%%%%%%%%%%%%%%%%%%%%%%%%%%%%%%%% 
%%%%%%%%%%%%%%%%%%%%%%%%%%%%%%%%%%%%%%%%%%%%%%%%%%%%%%%%%%%%%%%%%%%%%%%%%%%%%%% 
%%%%%%%%%%%%%%%%%%%%%%%%%%%%%%%%%%%%%%%%%%%%%%%%%%%%%%%%%%%%%%%%%%%%%%%%%%%%%%%


\begin{frame}[fragile]{\insertsection: Entorno \bftt{verbatim}}
  \begin{itemize}
  \item El texto encerrado entre \cmdbegin{verbatim} y
    \cmdend{verbatim} se escribirá directamente, con todos los saltos
    de línea y espacios, sin ejecutar ninguna orden \LaTeX.\\[1ex]
    \begin{exampletwouptiny}
\begin{verbatim}
#include<stdio.h>

int main()
{
  printf("Hello World");
  return 0;
}
\end{verbatim}
    \end{exampletwouptiny}
    \vskip 2ex
  \item Dentro de un párrafo, un comportamiento similar se puede 
    obtener con \cmdbs{verb}+text+.
  \end{itemize}
\end{frame}

%%%%%%%%%%%%%%%%%%%%%%%%%%%%%%%%%%%%%%%%%%%%%%%%%%%%%%%%%%%%%%%%%%%%%%%%%%%%%%% 
%%%%%%%%%%%%%%%%%%%%%%%%%%%%%%%%%%%%%%%%%%%%%%%%%%%%%%%%%%%%%%%%%%%%%%%%%%%%%%% 
%%%%%%%%%%%%%%%%%%%%%%%%%%%%%%%%%%%%%%%%%%%%%%%%%%%%%%%%%%%%%%%%%%%%%%%%%%%%%%%


\begin{frame}[fragile]{\insertsection: Paquetes \bftt{verbatim} y \bftt{fancyvrb}}
  \begin{itemize}
  \item El paquete \bftt{verbatim} nos permite incluir un fichero de
    texto como si estuviera dentro de un entorno \bftt{verbatim}.\\[2ex]
    \begin{exampletwouptiny}
\verbatiminput{main.c}
    \end{exampletwouptiny}
  \end{itemize}
\end{frame}

%%%%%%%%%%%%%%%%%%%%%%%%%%%%%%%%%%%%%%%%%%%%%%%%%%%%%%%%%%%%%%%%%%%%%%%%%%%%%%% 
%%%%%%%%%%%%%%%%%%%%%%%%%%%%%%%%%%%%%%%%%%%%%%%%%%%%%%%%%%%%%%%%%%%%%%%%%%%%%%% 
%%%%%%%%%%%%%%%%%%%%%%%%%%%%%%%%%%%%%%%%%%%%%%%%%%%%%%%%%%%%%%%%%%%%%%%%%%%%%%%


\begin{frame}[fragile]{\insertsection: Paquetes \bftt{verbatim} y \bftt{fancyvrb}}
  \begin{itemize}
  \item {\small El paquete \bftt{fancyvrb} nos permite incluir un fichero de
      texto como si estuviera dentro de un entorno \bftt{verbatim}.}\\[2ex]
  \item {\small Con el paquete \bftt{fancyvrb} se puede realizar
      tareas comunes a código-fuente, tales como: cambiar la fuente
      del texto y tamaño, numerar las líneas, etc.}\\[2ex]
    \begin{exampletwouptiny}
\VerbatimInput[frame=lines,
fontshape=sl,
fontsize=\scriptsize,
numbers=left,
formatcom=\color{blue}]
{main.c}
    \end{exampletwouptiny}

  \end{itemize}
\end{frame}

%%%%%%%%%%%%%%%%%%%%%%%%%%%%%%%%%%%%%%%%%%%%%%%%%%%%%%%%%%%%%%%%%%%%%%%%%%%%%%% 
%%%%%%%%%%%%%%%%%%%%%%%%%%%%%%%%%%%%%%%%%%%%%%%%%%%%%%%%%%%%%%%%%%%%%%%%%%%%%%% 
%%%%%%%%%%%%%%%%%%%%%%%%%%%%%%%%%%%%%%%%%%%%%%%%%%%%%%%%%%%%%%%%%%%%%%%%%%%%%%%


\begin{frame}[fragile]{\insertsection: Paquete \bftt{listings}}
  \begin{itemize}
  \item {\small El paquete \bftt{listings} se utiliza para imprimir
      código-fuente en \LaTeX{}. El entorno es similar al paquete 
      \bftt{verbatim}}.
    \begin{exampletwouptiny}
\begin{lstlisting}
  #include<stdio.h>
  
  int main()
  {
    printf("Hello World");
    return 0;
  }
\end{lstlisting}
    \end{exampletwouptiny}
    \vskip 2ex
  \item {\small Para personalizar el entorno \bftt{listings} se
      utiliza el comando \cmdbs{lstset}. El código siguiente debe ser
      agregado en el preámbulo del archivo \LaTeX{}.}
    \begin{minted}[fontsize=\scriptsize,frame=single]{latex}
\usepackage{listings}
\lstset{
  basicstyle=\tiny, language=c, fancyvrb=false, numbers=left,
  keywordstyle=\color{blue}\bfseries, frame=shadowbox,
  morekeywords={printf}, breaklines=true, 
  rulesepcolor=\color{blue}, stringstyle=\ttfamily}
    \end{minted}
  \end{itemize}
\end{frame}

%%%%%%%%%%%%%%%%%%%%%%%%%%%%%%%%%%%%%%%%%%%%%%%%%%%%%%%%%%%%%%%%%%%%%%%%%%%%%%% 
%%%%%%%%%%%%%%%%%%%%%%%%%%%%%%%%%%%%%%%%%%%%%%%%%%%%%%%%%%%%%%%%%%%%%%%%%%%%%%% 
%%%%%%%%%%%%%%%%%%%%%%%%%%%%%%%%%%%%%%%%%%%%%%%%%%%%%%%%%%%%%%%%%%%%%%%%%%%%%%%


\begin{frame}[fragile]{\insertsection: Paquete \bftt{minted}}
  \begin{itemize}
  \item {\small El paquete \bftt{minted} permite insertar código
      fuente y resaltar las la sintaxis del lenguaje utilizado. Este
      paquete utiliza una librería del lenguaje Python
      (\bftt{python-pygments}).}
    \vskip 2ex
    \begin{exampletwouptiny}
\begin{minted}[fontsize=\tiny,
  frame=single,
  linenos=true]{c}
  #include<stdio.h>
  
  int main()
  {
    printf("Hello World");
    return 0;
  }
\end{minted}
    \end{exampletwouptiny}
    \vskip 2ex
  \item {\small Los últimos dos paquetes, \bftt{listings} y
      \bftt{minted}, requieren otros paquetes \LaTeX{} (y en el caso de
      \bftt{minted} paquetes externos) para poder funcionar
      correctamente. Se aconseja leer las respectivas documentaciones de
      la página \href{}{http://ctan.org/}}
  \end{itemize}
\end{frame}

%%%%%%%%%%%%%%%%%%%%%%%%%%%%%%%%%%%%%%%%%%%%%%%%%%%%%%%%%%%%%%%%%%%%%%%%%%%%%%%
%%%%%%%%%%%%%%%%%%%%%%%%%%%%%%%%%%%%%%%%%%%%%%%%%%%%%%%%%%%%%%%%%%%%%%%%%%%%%%%
%%%%%%%%%%%%%%%%%%%%%%%%%%%%%%%%%%%%%%%%%%%%%%%%%%%%%%%%%%%%%%%%%%%%%%%%%%%%%%%

\section{Figuras (continuación)}


%%%%%%%%%%%%%%%%%%%%%%%%%%%%%%%%%%%%%%%%%%%%%%%%%%%%%%%%%%%%%%%%%%%%%%%%%%%%%%%
%%%%%%%%%%%%%%%%%%%%%%%%%%%%%%%%%%%%%%%%%%%%%%%%%%%%%%%%%%%%%%%%%%%%%%%%%%%%%%%
%%%%%%%%%%%%%%%%%%%%%%%%%%%%%%%%%%%%%%%%%%%%%%%%%%%%%%%%%%%%%%%%%%%%%%%%%%%%%%%
\begin{frame}[fragile]{\insertsection: Subfiguras}
  \begin{itemize}
  \item Anteriormente
    (\href{https://raw.github.com/guanucoluis/latex-course/master/es/part2.pdf}{Parte
      2}) vimos cómo insertar imágenes en \LaTeX{}. En esta parte trataremos
    casos específicos de la inserción de figuras.
  \item En el siguiente ejemplo tenemos dos imágenes que se encuentran
    vinculadas entre sí.
    \begin{minipage}{0.5\linewidth}
      \inputminted[fontsize=\tiny,frame=single,resetmargins]{latex}%
      {multiple-figures.tex}
    \end{minipage}
    \begin{minipage}{0.4\linewidth}
      % trim: l b r t
      \includegraphics[width=\textwidth,clip,trim=1in 1.1in 1in 1.1in,page=4]{multiple-figures.pdf}   
    \end{minipage}
    \end{itemize}
\end{frame}

%%%%%%%%%%%%%%%%%%%%%%%%%%%%%%%%%%%%%%%%%%%%%%%%%%%%%%%%%%%%%%%%%%%%%%%%%%%%%%%
%%%%%%%%%%%%%%%%%%%%%%%%%%%%%%%%%%%%%%%%%%%%%%%%%%%%%%%%%%%%%%%%%%%%%%%%%%%%%%%
%%%%%%%%%%%%%%%%%%%%%%%%%%%%%%%%%%%%%%%%%%%%%%%%%%%%%%%%%%%%%%%%%%%%%%%%%%%%%%%
\begin{frame}[fragile]{\insertsection: Subfiguras}
  \begin{itemize}
  \item Para mejorar la hoja anterior se usan los paquetes
    \bftt{caption} y \bftt{subcaption}. De esta forma se pueden agregar
    sub-flotantes en un único flotante.\\
    \begin{minipage}{0.5\linewidth}
      \inputminted[fontsize=\tiny,frame=single,resetmargins]{latex}%
      {subfigure-example.tex}
    \end{minipage}
    \begin{minipage}{0.4\linewidth}
      % trim: l b r t
      \includegraphics[width=\textwidth,clip,trim=1in 1.1in 1in 1.1in,page=4]{subfigure-example.pdf}   
    \end{minipage}
    \end{itemize}
\end{frame}

%%%%%%%%%%%%%%%%%%%%%%%%%%%%%%%%%%%%%%%%%%%%%%%%%%%%%%%%%%%%%%%%%%%%%%%%%%%%%%% 
%%%%%%%%%%%%%%%%%%%%%%%%%%%%%%%%%%%%%%%%%%%%%%%%%%%%%%%%%%%%%%%%%%%%%%%%%%%%%%%
%%%%%%%%%%%%%%%%%%%%%%%%%%%%%%%%%%%%%%%%%%%%%%%%%%%%%%%%%%%%%%%%%%%%%%%%%%%%%%%
\section{Bibliografía (continuación)}

%%%%%%%%%%%%%%%%%%%%%%%%%%%%%%%%%%%%%%%%%%%%%%%%%%%%%%%%%%%%%%%%%%%%%%%%%%%%%%%
%%%%%%%%%%%%%%%%%%%%%%%%%%%%%%%%%%%%%%%%%%%%%%%%%%%%%%%%%%%%%%%%%%%%%%%%%%%%%%%
%%%%%%%%%%%%%%%%%%%%%%%%%%%%%%%%%%%%%%%%%%%%%%%%%%%%%%%%%%%%%%%%%%%%%%%%%%%%%%%
\begin{frame}[fragile]
  \frametitle{\insertsection: El entorno \bftt{thebibliography}}
  \begin{itemize}
  \item En la
    \href{https://raw.github.com/guanucoluis/latex-course/master/es/part2.pdf}{Parte
      2} se mostró como utilizar bases de datos de bibliografías para
    nuestros documentos \LaTeX{}. Pero para el caso de querer generar
    un simple reporte, el proceso de compilación con \bftt{bibtex}
    resulta lento.
  \item \LaTeX{} provee un entorno llamado \bftt{thebibliography}. De
    esta forma se puede agregar bibliografía en nuestro documento sin
    la necesidad de llamar a \bftt{bibtex}.
    \vskip 2ex
    \begin{minted}[fontsize=\small,frame=single]{latex}
\begin{thebibliography}{1}

\bibitem{lamport94}
  Leslie Lamport,
  \emph{\LaTeX: a document preparation system},
  Addison Wesley, Massachusetts,
  2nd edition,
  1994.

\end{thebibliography}
    \end{minted}
  \end{itemize}
\end{frame}

%%%%%%%%%%%%%%%%%%%%%%%%%%%%%%%%%%%%%%%%%%%%%%%%%%%%%%%%%%%%%%%%%%%%%%%%%%%%%%%
%%%%%%%%%%%%%%%%%%%%%%%%%%%%%%%%%%%%%%%%%%%%%%%%%%%%%%%%%%%%%%%%%%%%%%%%%%%%%%%
%%%%%%%%%%%%%%%%%%%%%%%%%%%%%%%%%%%%%%%%%%%%%%%%%%%%%%%%%%%%%%%%%%%%%%%%%%%%%%%
\begin{frame}[fragile]{\insertsection: El entorno \bftt{thebibliography}}
  \begin{itemize}
  \item A continuación se muestra el mismo ejemplo utilizado con
    \bftt{bibtex}.
    \vskip 2ex
    \begin{minipage}{0.5\linewidth}
      \inputminted[fontsize=\tiny,frame=single,resetmargins]{latex}%
      {biblio-example.tex}
    \end{minipage}
    \begin{minipage}{0.4\linewidth}
      % trim: l b r t
      \includegraphics[width=\textwidth,clip,trim=1in 1.1in 1in 1.1in]{biblio-example.pdf}   
    \end{minipage}
    \end{itemize}
\end{frame}

%%%%%%%%%%%%%%%%%%%%%%%%%%%%%%%%%%%%%%%%%%%%%%%%%%%%%%%%%%%%%%%%%%%%%%%%%%%%%%
%%%%%%%%%%%%%%%%%%%%%%%%%%%%%%%%%%%%%%%%%%%%%%%%%%%%%%%%%%%%%%%%%%%%%%%%%%%%%%
%%%%%%%%%%%%%%%%%%%%%%%%%%%%%%%%%%%%%%%%%%%%%%%%%%%%%%%%%%%%%%%%%%%%%%%%%%%%%%
\section{Dibujando con circui\protect\tikzname}

%%%%%%%%%%%%%%%%%%%%%%%%%%%%%%%%%%%%%%%%%%%%%%%%%%%%%%%%%%%%%%%%%%%%%%%%%%%%%%%
%%%%%%%%%%%%%%%%%%%%%%%%%%%%%%%%%%%%%%%%%%%%%%%%%%%%%%%%%%%%%%%%%%%%%%%%%%%%%%%
%%%%%%%%%%%%%%%%%%%%%%%%%%%%%%%%%%%%%%%%%%%%%%%%%%%%%%%%%%%%%%%%%%%%%%%%%%%%%%%
\begin{frame}[fragile]{\insertsection}
  \begin{itemize}
    \item El paquete circui\tikzname{} provee macros para componer
      diagramas eléctricos/electrónicos  en \LaTeX{}.
    \item Solo debemos cargar el paquete \bftt{circuitikz}. Éste
      cargará automáticamente el paquete \bftt{tikz}.
    \vskip 2ex
    \begin{exampletwouptiny}
\ctikzset{bipoles/length=1cm}
\begin{circuitikz}
  [scale=0.9]
  \draw[help lines, green]
  (0,0) grid (2,2);
  \draw 
  (0,0) to[sV=5<\volt>]
  (0,2) to[R=47<\ohm>]
  (2,2) to[L=5<\milli\henry>]
  (2,0) to[C=10<\micro\farad>]
  (0,0);
\end{circuitikz}
    \end{exampletwouptiny}
  \end{itemize}
\end{frame}

%%%%%%%%%%%%%%%%%%%%%%%%%%%%%%%%%%%%%%%%%%%%%%%%%%%%%%%%%%%%%%%%%%%%%%%%%%%%%%%
%%%%%%%%%%%%%%%%%%%%%%%%%%%%%%%%%%%%%%%%%%%%%%%%%%%%%%%%%%%%%%%%%%%%%%%%%%%%%%%
%%%%%%%%%%%%%%%%%%%%%%%%%%%%%%%%%%%%%%%%%%%%%%%%%%%%%%%%%%%%%%%%%%%%%%%%%%%%%%%
\begin{frame}[fragile]{\insertsection}
  \begin{itemize}
  \item Al igual que muchos paquetes que hemos utilizado,
    circui\tikzname{} permite recibir opciones para personalizar su uso.
    \vskip 2ex
    \begin{minted}[fontsize=\footnotesize,frame=single]{latex}
\usepackage[american,cuteinductors,siunitx]{circuitikz}
\usepackage{siunitx}
    \end{minted}
  \item En función de los argumentos opcionales del ejemplo anterior tenemos:
    \begin{description}
    \item[\bftt{american}] define qué simbología se utilizará. En este
      caso las estándares americanas.
    \item[\bftt{cuteinductors}] permite que el símbolo del inductor
      se encuentre más ondulado que el estándar.
    \item[\bftt{siunitx}] informa a \tikzname{} que utilizará las
      definiciones de unidades desde el paquete \bftt{siunitx}.
    \end{description}
  \end{itemize}
\end{frame}

%%%%%%%%%%%%%%%%%%%%%%%%%%%%%%%%%%%%%%%%%%%%%%%%%%%%%%%%%%%%%%%%%%%%%%%%%%%%%%%
%%%%%%%%%%%%%%%%%%%%%%%%%%%%%%%%%%%%%%%%%%%%%%%%%%%%%%%%%%%%%%%%%%%%%%%%%%%%%%%
%%%%%%%%%%%%%%%%%%%%%%%%%%%%%%%%%%%%%%%%%%%%%%%%%%%%%%%%%%%%%%%%%%%%%%%%%%%%%%%
\begin{frame}[fragile]{\insertsection: Tipos de Componentes}
  \begin{itemize}
  \item Monopolares\\
    \begin{exampletwouptiny}
\begin{circuitikz}
  \draw node[ground]{GND};
\end{circuitikz}
    \end{exampletwouptiny}
  \item Bipolares\\
    \begin{exampletwouptiny}
\begin{circuitikz} 
\ctikzset{bipoles/length=1cm}
\draw (0,0) to
[R=$R_1$,i=$i_1$,v=$v_1$,o-*] (2,0);
\end{circuitikz}
    \end{exampletwouptiny}
  \item Tripolar\\
    \begin{exampletwouptiny}
\begin{circuitikz}
  \draw (0,0) node[npn](npn){T1}
  (npn.base) node[anchor=east]{B}
  (npn.collector) node[anchor=south]{C}
  (npn.emitter) node[anchor=north]{E};
\end{circuitikz}
    \end{exampletwouptiny}
  \end{itemize}
\end{frame}


%%%%%%%%%%%%%%%%%%%%%%%%%%%%%%%%%%%%%%%%%%%%%%%%%%%%%%%%%%%%%%%%%%%%%%%%%%%%%%%
%%%%%%%%%%%%%%%%%%%%%%%%%%%%%%%%%%%%%%%%%%%%%%%%%%%%%%%%%%%%%%%%%%%%%%%%%%%%%%%
%%%%%%%%%%%%%%%%%%%%%%%%%%%%%%%%%%%%%%%%%%%%%%%%%%%%%%%%%%%%%%%%%%%%%%%%%%%%%%%
\begin{frame}[fragile]{\insertsection: Tipos de Componentes}
  \begin{itemize}
  \item Doble-bipolares\\[1ex]
    \begin{exampletwouptiny}
\begin{circuitikz} \draw 
  (0,0) node[transformer](T){}
  (T.A1) node[anchor=east] {A1}
  (T.A2) node[anchor=east] {A2}
  (T.B1) node[anchor=west] {B1}
  (T.B2) node[anchor=west] {B2}
  (T.base) node{N=$\frac{1}{20}$};
\end{circuitikz}
    \end{exampletwouptiny}
  \item Compuertas lógicas\\[1ex]
    \begin{exampletwouptiny}
\begin{circuitikz} \draw 
  (0,0) node[and port](myand) {}
  (myand.in 1) node[anchor=east] {A}
  (myand.in 2) node[anchor=east] {B}
  (myand.out) node[anchor=west] {O};
  \draw (myand) 
  node[below=6mm]{$O = A \cdot B$};
\end{circuitikz}
    \end{exampletwouptiny}
  \end{itemize}
\end{frame}


%%%%%%%%%%%%%%%%%%%%%%%%%%%%%%%%%%%%%%%%%%%%%%%%%%%%%%%%%%%%%%%%%%%%%%%%%%%%%%%
%%%%%%%%%%%%%%%%%%%%%%%%%%%%%%%%%%%%%%%%%%%%%%%%%%%%%%%%%%%%%%%%%%%%%%%%%%%%%%%
%%%%%%%%%%%%%%%%%%%%%%%%%%%%%%%%%%%%%%%%%%%%%%%%%%%%%%%%%%%%%%%%%%%%%%%%%%%%%%%
\begin{frame}[fragile]{\insertsection: Tipos de Componentes}
  \begin{itemize}
  \item Amplificadores
    \begin{itemize}
    \item Operacionales\\[1ex]
    \begin{exampletwouptiny}
\begin{circuitikz} \draw 
  (0,0) node[op amp] (opamp){}
  (opamp.+) node[left]{$v_+$}
  (opamp.-) node[left]{$v_-$}
  (opamp.out) node[right]{$v_o$}
  (opamp.down) node[ground]{}
  (opamp.up) ++ (0,.5) node[above]
  {\SI{12}{\volt}} -- (opamp.up);
\end{circuitikz}
    \end{exampletwouptiny}
    \item \textsl{Buffers}\\[1ex]
    \begin{exampletwouptiny}
\begin{circuitikz} \draw 
  (0,0) node[buffer] (buf){}
  (buf.in) node[left]{In}
  (buf.out) node[right]{Out};
\end{circuitikz}
    \end{exampletwouptiny}
    \end{itemize}
  \end{itemize}
\end{frame}

%%%%%%%%%%%%%%%%%%%%%%%%%%%%%%%%%%%%%%%%%%%%%%%%%%%%%%%%%%%%%%%%%%%%%%%%%%%%%%%
%%%%%%%%%%%%%%%%%%%%%%%%%%%%%%%%%%%%%%%%%%%%%%%%%%%%%%%%%%%%%%%%%%%%%%%%%%%%%%%
%%%%%%%%%%%%%%%%%%%%%%%%%%%%%%%%%%%%%%%%%%%%%%%%%%%%%%%%%%%%%%%%%%%%%%%%%%%%%%%
\begin{frame}[fragile]{\insertsection: Integración con \bftt{siunitx}}
  \begin{itemize}
  \item Existen dos modos de utilizar las unidades con el paquete
    \bftt{siunitx}\\[1ex]
    \begin{exampletwouptiny}
\begin{circuitikz} \draw 
  (0,0) to[R, l=1<\kilo\ohm>]
  (2,0);
\end{circuitikz}
    \end{exampletwouptiny}
    \vskip 2ex
    \begin{exampletwouptiny}
\begin{circuitikz} \draw 
  (0,0) to[R, l=\SI{1}{\kilo\ohm}]
  (2,0);
\end{circuitikz}
    \end{exampletwouptiny}
  \end{itemize}
\end{frame}


%%%%%%%%%%%%%%%%%%%%%%%%%%%%%%%%%%%%%%%%%%%%%%%%%%%%%%%%%%%%%%%%%%%%%%%%%%%%%%%
%%%%%%%%%%%%%%%%%%%%%%%%%%%%%%%%%%%%%%%%%%%%%%%%%%%%%%%%%%%%%%%%%%%%%%%%%%%%%%%
%%%%%%%%%%%%%%%%%%%%%%%%%%%%%%%%%%%%%%%%%%%%%%%%%%%%%%%%%%%%%%%%%%%%%%%%%%%%%%%
\begin{frame}[fragile]{\insertsection: Espejado de componentes}
  \begin{itemize}
  \item En el momento de instanaciar un componente, se puede
    especificar que el símbolo sea espejado.\\[1ex]
    \begin{exampletwouptiny}
\begin{circuitikz} \draw 
  (0,0) to[ospst=T,v=v,i=$i_1$]
  (2,0);
\end{circuitikz}
    \end{exampletwouptiny}
    \vskip 2ex
    \begin{exampletwouptiny}
\begin{circuitikz} \draw 
  (0,0) to[ospst=T,mirror,v=v,i=$i_1$]
  (2,0);
\end{circuitikz}
    \end{exampletwouptiny}
  \item Tener en cuenta que el \bftt{label} también se modifica. Pero
    esto no sucede en el caso de la indicación de corriente.
    \vskip 2ex
    \begin{exampletwouptiny}
\begin{circuitikz} \draw 
  (0,0) to[ospst=T,mirror,v=v,i=$i_1$]
  (2,0);
\end{circuitikz}
    \end{exampletwouptiny}
  \end{itemize}
\end{frame}

%%%%%%%%%%%%%%%%%%%%%%%%%%%%%%%%%%%%%%%%%%%%%%%%%%%%%%%%%%%%%%%%%%%%%%%%%%%%%%%
%%%%%%%%%%%%%%%%%%%%%%%%%%%%%%%%%%%%%%%%%%%%%%%%%%%%%%%%%%%%%%%%%%%%%%%%%%%%%%%
%%%%%%%%%%%%%%%%%%%%%%%%%%%%%%%%%%%%%%%%%%%%%%%%%%%%%%%%%%%%%%%%%%%%%%%%%%%%%%%
\begin{frame}[fragile]{\insertsection: Parámetros de Circui\tikzname{}}
  \begin{itemize}
  \item La mayoría de los macros que provee circui\tikzname{} hacen
    uso de los \bftt{pgfkeys} para la configuración de las imágenes
    que ofrece. Recuerde que Circui\tikzname{} utiliza el paquete
    \tikzname{}.
  \item Para manipular las macros utilizamos el comando \cmdbs{ctikzset}.\\[1ex]
    \begin{exampletwouptiny}
\tikz \draw (0,0) to[R=1<\ohm>](2,0); 
\par
\ctikzset{bipoles/resistor/height=.6}
\tikz \draw (0,0) to[R=1<\ohm>](2,0);
    \end{exampletwouptiny}
    \vskip 2ex
    \begin{exampletwouptiny}
\tikz \draw 
(0,0) to[C, i=1<\milli\ampere>] (2,0); 
\par
\ctikzset{current/distance=.8}
\tikz \draw 
(0,0) to[C, i=1<\milli\ampere>] (2,0); 
    \end{exampletwouptiny}
  \end{itemize}
\end{frame}

%%%%%%%%%%%%%%%%%%%%%%%%%%%%%%%%%%%%%%%%%%%%%%%%%%%%%%%%%%%%%%%%%%%%%%%%%%%%%%%
%%%%%%%%%%%%%%%%%%%%%%%%%%%%%%%%%%%%%%%%%%%%%%%%%%%%%%%%%%%%%%%%%%%%%%%%%%%%%%%
%%%%%%%%%%%%%%%%%%%%%%%%%%%%%%%%%%%%%%%%%%%%%%%%%%%%%%%%%%%%%%%%%%%%%%%%%%%%%%%
\begin{frame}[fragile]{\insertsection: Tamaño de los componentes}
  \begin{itemize}
  \item Tal vez el parámetro más importante sea
    \cmdbs{circuitikzbasekey/bipoles/length}, que puede considerarse
    como el largo de un resistor; todas las demás longitudes serán
    relativa a este \bftt{key}.\\[1ex]
    \begin{minted}[fontsize=\footnotesize,frame=single]{latex}
\ctikzset{bipoles/length=1.4cm}
\begin{circuitikz}[scale=1.2]\draw
...
    \end{minted}
    \vskip 2ex
\ctikzset{bipoles/length=1.4cm}
\begin{center}
\begin{circuitikz}[scale=1.2]\draw
  (0,0) node[anchor=east] {B}
  to[short , o-*] (1,0)
  to[R=20<\ohm>, *-*] (1,2)
  to[R=10<\ohm>, v=$v_x$] (3,2) -- (4,2)
  to[ cI=$\frac{\si{\siemens}}{5} v_x$, *-*] (4,0) -- (3,0)
  to[R=5<\ohm>, *-*] (3,2)
  (3,0) -- (1,0)
  (1,2) to[short , -o] (0,2) node[anchor=east]{A};
\end{circuitikz}
\end{center}
  \end{itemize}
\end{frame}

%%%%%%%%%%%%%%%%%%%%%%%%%%%%%%%%%%%%%%%%%%%%%%%%%%%%%%%%%%%%%%%%%%%%%%%%%%%%%%%
%%%%%%%%%%%%%%%%%%%%%%%%%%%%%%%%%%%%%%%%%%%%%%%%%%%%%%%%%%%%%%%%%%%%%%%%%%%%%%%
%%%%%%%%%%%%%%%%%%%%%%%%%%%%%%%%%%%%%%%%%%%%%%%%%%%%%%%%%%%%%%%%%%%%%%%%%%%%%%%
\begin{frame}[fragile]{\insertsection: Tamaño de los componentes}
  \begin{itemize}
  \item Tal vez el parámetro más importante sea
    \cmdbs{circuitikzbasekey/bipoles/length}, que puede considerarse
    como el largo de un resistor; todas las demás longitudes serán
    relativa a este \bftt{key}.\\[1ex]
    \begin{minted}[fontsize=\footnotesize,frame=single]{latex}
\ctikzset{bipoles/length=.8cm}
\begin{circuitikz}[scale=1.2]\draw
...
    \end{minted}
    \vskip 2ex
\ctikzset{bipoles/length=.8cm}
\begin{center}
\begin{circuitikz}[scale=1.2]\draw
  (0,0) node[anchor=east] {B}
  to[short , o-*] (1,0)
  to[R=20<\ohm>, *-*] (1,2)
  to[R=10<\ohm>, v=$v_x$] (3,2) -- (4,2)
  to[ cI=$\frac{\si{\siemens}}{5} v_x$, *-*] (4,0) -- (3,0)
  to[R=5<\ohm>, *-*] (3,2)
  (3,0) -- (1,0)
  (1,2) to[short , -o] (0,2) node[anchor=east]{A};
\end{circuitikz}
\end{center}
  \end{itemize}
\end{frame}


%%%%%%%%%%%%%%%%%%%%%%%%%%%%%%%%%%%%%%%%%%%%%%%%%%%%%%%%%%%%%%%%%%%%%%%%%%%%%%%
%%%%%%%%%%%%%%%%%%%%%%%%%%%%%%%%%%%%%%%%%%%%%%%%%%%%%%%%%%%%%%%%%%%%%%%%%%%%%%%
%%%%%%%%%%%%%%%%%%%%%%%%%%%%%%%%%%%%%%%%%%%%%%%%%%%%%%%%%%%%%%%%%%%%%%%%%%%%%%%
\begin{frame}[fragile]{\insertsection: Varios ejemplos}
  \begin{itemize}
  \item Revise
    \href{http://texample.net}{\TeX{}ample.net} para
    muchos ejemplos de \fbox{\href{http://texample.net/tikz/examples/tag/circuitikz/}{Circui\tikzname{}}}:
  \end{itemize}
  \begin{figure}
    \href{http://texample.net/tikz/examples/mosfet/}{%
      \includegraphics[width=0.3\textwidth]{mosfet}}
    \href{http://texample.net/tikz/examples/collpits/}{%
      \includegraphics[width=0.3\textwidth]{collpits}}
    \href{http://texample.net/tikz/examples/4-bit-counter/}{%
      \includegraphics[width=0.3\textwidth]{4-bit-counter}}
  \end{figure}
\end{frame}

%%%%%%%%%%%%%%%%%%%%%%%%%%%%%%%%%%%%%%%%%%%%%%%%%%%%%%%%%%%%%%%%%%%%%%%%%%%%%%
%%%%%%%%%%%%%%%%%%%%%%%%%%%%%%%%%%%%%%%%%%%%%%%%%%%%%%%%%%%%%%%%%%%%%%%%%%%%%%
%%%%%%%%%%%%%%%%%%%%%%%%%%%%%%%%%%%%%%%%%%%%%%%%%%%%%%%%%%%%%%%%%%%%%%%%%%%%%%
\section{Compilación bajo Consola}

%%%%%%%%%%%%%%%%%%%%%%%%%%%%%%%%%%%%%%%%%%%%%%%%%%%%%%%%%%%%%%%%%%%%%%%%%%%%%%%
%%%%%%%%%%%%%%%%%%%%%%%%%%%%%%%%%%%%%%%%%%%%%%%%%%%%%%%%%%%%%%%%%%%%%%%%%%%%%%%
%%%%%%%%%%%%%%%%%%%%%%%%%%%%%%%%%%%%%%%%%%%%%%%%%%%%%%%%%%%%%%%%%%%%%%%%%%%%%%%
\begin{frame}[fragile]{\insertsection: Flujo de compilación}
  \begin{block}{}
  Tanto la plataforma \href{https://www.overleaf.com}{Overleaf}
  y el software \href{http://www.xm1math.net/texmaker/}{TexMaker} nos
  permiten obtener una salida PDF a partir de texto plano
  (archivo.tex). Pero en ningún momento intervenimos en el proceso de
  \emph{compilación}.    
  \end{block}
  \vskip 2ex
  \begin{center}
    \includegraphics[width=\textwidth]{texmakertop_big}
  \end{center}    
\end{frame}

%%%%%%%%%%%%%%%%%%%%%%%%%%%%%%%%%%%%%%%%%%%%%%%%%%%%%%%%%%%%%%%%%%%%%%%%%%%%%%%
%%%%%%%%%%%%%%%%%%%%%%%%%%%%%%%%%%%%%%%%%%%%%%%%%%%%%%%%%%%%%%%%%%%%%%%%%%%%%%%
%%%%%%%%%%%%%%%%%%%%%%%%%%%%%%%%%%%%%%%%%%%%%%%%%%%%%%%%%%%%%%%%%%%%%%%%%%%%%%%
\begin{frame}[fragile]{\insertsection: Flujo de compilación}
  
  \begin{block}{}
    Para independizarse de cualquier plataforma o interfaz, se
    presenta el flujo de compilación que nos permite alcanzar un
    archivo de salida a partir de texto fuente \LaTeX{}.
  \end{block}
  \vskip 2ex
  \begin{figure}[h]
    \centering
    \includegraphics[width=\textwidth]{latex-flow}
    \caption{Basado en la imagen
\href{http://commons.wikimedia.org/wiki/File:LaTeX\_diagram.svg\#/media/File:LaTeX\_diagram.svg}{``LaTeX
  diagram''} de Alessio Damato. Publicado bajo licencia \href{http://creativecommons.org/licenses/by-sa/3.0/}{CC BY-SA 3.0}.}
  \end{figure}
%  \begin{center}
%    \includegraphics[width=\textwidth]{latex-flow}
%  \end{center}
\end{frame}

%%%%%%%%%%%%%%%%%%%%%%%%%%%%%%%%%%%%%%%%%%%%%%%%%%%%%%%%%%%%%%%%%%%%%%%%%%%%%%%
%%%%%%%%%%%%%%%%%%%%%%%%%%%%%%%%%%%%%%%%%%%%%%%%%%%%%%%%%%%%%%%%%%%%%%%%%%%%%%%
%%%%%%%%%%%%%%%%%%%%%%%%%%%%%%%%%%%%%%%%%%%%%%%%%%%%%%%%%%%%%%%%%%%%%%%%%%%%%%%
\begin{frame}[fragile]{\insertsection: pdf\LaTeX{}}
  \begin{itemize}
  \item Tanto la herramienta \bftt{latex} como \bftt{pdflatex}
    proporcionan un archivo de salida listo para ser impreso.
  \item El formato PDF es el más utilizado y por lo tanto se utilizará
    \bftt{pdflatex} para compilar nuestros archivos \LaTeX{}.
    \vskip 2ex
    \begin{center}
      \begin{minted}[fontsize=\tiny,frame=single]{rout}
luis@luis-laptop:$ pdflatex structure-title.tex 
This is pdfTeX, Version 3.1415926-2.4-1.40.13 (TeX Live 2012/Debian)
...
(./structure-title.tex
LaTeX2e <2011/06/27>
Babel <v3.8m> and hyphenation patterns for english, dumylang, nohyphenation, 
...
celandic, kurmanji, polish, portuguese, galician, catalan, spanish, swedish
loaded.
...
(/usr/share/texlive/texmf-dist/tex/latex/base/article.cls
Document Class: article 2007/10/19 v1.4h Standard LaTeX document class
(/usr/share/texlive/texmf-dist/tex/latex/base/size10.clo))
No file structure-title.aux.
[1{/var/lib/texmf/fonts/map/pdftex/updmap/pdftex.map}] (./structure-title.aux) 
)</usr/share/texlive/texmf-dist/fonts/type1/public/amsfonts/cm/cmbx9.pfb></usr/
share/texlive/texmf-dist/fonts/type1/public/amsfonts/cm/cmr10.pfb></usr/share/t
exlive/texmf-dist/fonts/type1/public/amsfonts/cm/cmr12.pfb></usr/share/texlive/
texmf-dist/fonts/type1/public/amsfonts/cm/cmr17.pfb></usr/share/texlive/texmf-d
ist/fonts/type1/public/amsfonts/cm/cmr9.pfb>
Output written on structure-title.pdf (1 page, 44792 bytes).
Transcript written on structure-title.log.
      \end{minted}
    \end{center}
  \end{itemize}
\end{frame}

%%%%%%%%%%%%%%%%%%%%%%%%%%%%%%%%%%%%%%%%%%%%%%%%%%%%%%%%%%%%%%%%%%%%%%%%%%%%%%%
%%%%%%%%%%%%%%%%%%%%%%%%%%%%%%%%%%%%%%%%%%%%%%%%%%%%%%%%%%%%%%%%%%%%%%%%%%%%%%%
%%%%%%%%%%%%%%%%%%%%%%%%%%%%%%%%%%%%%%%%%%%%%%%%%%%%%%%%%%%%%%%%%%%%%%%%%%%%%%%
\begin{frame}[fragile]{\insertsection: pdf\LaTeX{}}
  \begin{itemize}
  \item \bftt{pdflatex} dispone de varias opciones. Para obtener un
    resumen de éstas se puede lanzar el comando \bftt{pdflatex --help}.
    \vskip 1ex
    \begin{center}
      \begin{minted}[fontsize=\tiny,frame=single]{rout}
luis@luis-laptop:$ pdflatex --help
Usage: pdftex [OPTION]... [TEXNAME[.tex]] [COMMANDS]
   or: pdftex [OPTION]... \FIRST-LINE
   or: pdftex [OPTION]... &FMT ARGS
...
  If no arguments or options are specified, prompt for input.

-draftmode              switch on draft mode (generates no output PDF)
[-no]-file-line-error   disable/enable file:line:error style messages
-halt-on-error          stop processing at the first error
-jobname=STRING         set the job name to STRING
-output-comment=STRING  use STRING for DVI file comment instead of date
                          (no effect for PDF)
-output-directory=DIR   use existing DIR as the directory to write files in
-output-format=FORMAT   use FORMAT for job output; FORMAT is `dvi' or `pdf'
[-no]-shell-escape      disable/enable \write18{SHELL COMMAND}
-shell-restricted       enable restricted \write18
-src-specials           insert source specials into the DVI file
-8bit                   make all characters printable by default
-help                   display this help and exit
-version                output version information and exit
      \end{minted}
    \end{center}
  \end{itemize}
\end{frame}

%%%%%%%%%%%%%%%%%%%%%%%%%%%%%%%%%%%%%%%%%%%%%%%%%%%%%%%%%%%%%%%%%%%%%%%%%%%%%%%
%%%%%%%%%%%%%%%%%%%%%%%%%%%%%%%%%%%%%%%%%%%%%%%%%%%%%%%%%%%%%%%%%%%%%%%%%%%%%%%
%%%%%%%%%%%%%%%%%%%%%%%%%%%%%%%%%%%%%%%%%%%%%%%%%%%%%%%%%%%%%%%%%%%%%%%%%%%%%%%
\begin{frame}[fragile]{\insertsection: pdf\LaTeX{}}
  \begin{itemize}
  \item El proceso de compilación genera archivos temporales. Los más
    comunes son;
    \begin{description}{\scriptsize
      \item[\bftt{latexfile.aux}] archivo que conserva información de
        una compilación a la siguiente (por ejemplo, en las referencias
        cruzadas).
      \item[\bftt{latexfile.log}] registra información detallada de la
        compilación (por ejemplo, paquetes cargados, mensajes de warning y
        errores).}
    \end{description}
    \begin{center}
      \begin{minted}[fontsize=\tiny,frame=single]{rout}
luis@luis-laptop:$ ls -l
-rw-r--r-- 1 luis luis      8 may 17 22:05 structure-title.aux
-rw-r--r-- 1 luis luis   3762 may 17 22:05 structure-title.log
-rw-r--r-- 1 luis luis  44792 may 17 22:05 structure-title.pdf
-rw-r--r-- 1 luis luis    179 mar 28 16:06 structure-title.tex
      \end{minted}
    \end{center}
  \item Los archivos generados depende de la complejidad y demanda
    del archivo fuente. Sí estaríamos utilizando el comando
    \cmdbs{tableofcontents} seguramente tendríamos un archivo
    auxiliar \bftt{latexfile.toc}.
  \end{itemize}
\end{frame}

%%%%%%%%%%%%%%%%%%%%%%%%%%%%%%%%%%%%%%%%%%%%%%%%%%%%%%%%%%%%%%%%%%%%%%%%%%%%%%%
%%%%%%%%%%%%%%%%%%%%%%%%%%%%%%%%%%%%%%%%%%%%%%%%%%%%%%%%%%%%%%%%%%%%%%%%%%%%%%%
%%%%%%%%%%%%%%%%%%%%%%%%%%%%%%%%%%%%%%%%%%%%%%%%%%%%%%%%%%%%%%%%%%%%%%%%%%%%%%%
\begin{frame}[fragile]{\insertsection: pdf\LaTeX{}}
  \begin{itemize}
  \item La salida estándar de \bftt{pdflatex}, que se guarda en el
    archivo \bftt{.log}, proporciona información útil al editor. 
  \item Consideremos que estamos utilizando referencias cruzadas y es
    la primera vez que se compila. \bftt{pdflatex} le avisará que no
    encuentra las correspondientes referencias existentes.
    \begin{center}
      \begin{minted}[fontsize=\tiny,frame=single,linenos=true]{rout}
luis@luis-laptop:$ pdflatex structure-crossref.tex 
This is pdfTeX, Version 3.1415926-2.4-1.40.13 (TeX Live 2012/Debian)
 restricted \write18 enabled.
entering extended mode
(./structure-crossref.tex
LaTeX2e <2011/06/27>
...
No file structure-crossref.aux.

LaTeX Warning: Reference `sec:metodo' on page 1 undefined on input line 8.
LaTeX Warning: Reference `eq:euler' on page 1 undefined on input line 19.

[1{/var/lib/texmf/fonts/map/pdftex/updmap/pdftex.map}]
(./structure-crossref.aux)

LaTeX Warning: There were undefined references.
LaTeX Warning: Label(s) may have changed. Rerun to get cross-references right.
 )
...
Output written on structure-crossref.pdf (1 page, 47567 bytes).
Transcript written on structure-crossref.log.
      \end{minted}
    \end{center}
  \end{itemize}
\end{frame}

%%%%%%%%%%%%%%%%%%%%%%%%%%%%%%%%%%%%%%%%%%%%%%%%%%%%%%%%%%%%%%%%%%%%%%%%%%%%%%%
%%%%%%%%%%%%%%%%%%%%%%%%%%%%%%%%%%%%%%%%%%%%%%%%%%%%%%%%%%%%%%%%%%%%%%%%%%%%%%%
%%%%%%%%%%%%%%%%%%%%%%%%%%%%%%%%%%%%%%%%%%%%%%%%%%%%%%%%%%%%%%%%%%%%%%%%%%%%%%%
\begin{frame}[fragile]{\insertsection: pdf\LaTeX{}}
  \begin{itemize}
  \item Las líneas \texttt{10} y \texttt{11} de la salida estándar de
    la compilación son consecuencia de no contar con las referencias
    previamente generadas. La línea \texttt{8} informa que no existe
    una compilación previa.
  \item Si nuevamente compilamos, veremos una salida diferente.
    \begin{center}
      \begin{minted}[fontsize=\tiny,frame=single]{rout}
luis@luis-laptop:$ pdflatex structure-crossref.tex 
This is pdfTeX, Version 3.1415926-2.4-1.40.13 (TeX Live 2012/Debian)
 restricted \write18 enabled.
entering extended mode
(./structure-crossref.tex
LaTeX2e <2011/06/27>
...
(./structure-crossref.aux) [1{/var/lib/texmf/fonts/map/pdftex/updmap/pdftex.map
}] (./structure-crossref.aux) )
(see the transcript file for additional information)</usr/share/texlive/texmf-d
ist/fonts/type1/public/amsfonts/cm/cmbx12.pfb></usr/share/texlive/texmf-dist/fo
nts/type1/public/amsfonts/cm/cmmi10.pfb></usr/share/texlive/texmf-dist/fonts/ty
pe1/public/amsfonts/cm/cmmi7.pfb></usr/share/texlive/texmf-dist/fonts/type1/pub
lic/amsfonts/cm/cmr10.pfb>
Output written on structure-crossref.pdf (1 page, 38880 bytes).
Transcript written on structure-crossref.log.
      \end{minted}
    \end{center}
  \end{itemize}
\end{frame}
%$

%%%%%%%%%%%%%%%%%%%%%%%%%%%%%%%%%%%%%%%%%%%%%%%%%%%%%%%%%%%%%%%%%%%%%%%%%%%%%%%
%%%%%%%%%%%%%%%%%%%%%%%%%%%%%%%%%%%%%%%%%%%%%%%%%%%%%%%%%%%%%%%%%%%%%%%%%%%%%%%
%%%%%%%%%%%%%%%%%%%%%%%%%%%%%%%%%%%%%%%%%%%%%%%%%%%%%%%%%%%%%%%%%%%%%%%%%%%%%%%
\begin{frame}[fragile]{\insertsection: la herramienta \bftt{make}}
  \begin{itemize}
  \item Cuando se trabaja con proyectos complejos resulta tedioso el
    proceso de re-compilar una y otra vez los archivos fuentes.
  \item Sí sabemos los comandos que debemos utilizar, podríamos
    encontrar una herramienta que realice los pasos correctos hasta
    obtener nuestro documento final, o quizá eliminar los archivos
    auxiliares, etc.
    
    \begin{block}{\bftt{make}}
      {\scriptsize Software integrado a los sistemas operativos
      GNU/Linux que permite automatizar procesos. Una de las ventajas
      es que puede hacer uso de todos los programas con los que cuenta
      el sistema operativo.}
    \end{block}

  \item \bftt{make} lee por defecto un archivo \bftt{Makefile} en el
    directorio donde se ejecuta. 
  \end{itemize}
\end{frame}

%%%%%%%%%%%%%%%%%%%%%%%%%%%%%%%%%%%%%%%%%%%%%%%%%%%%%%%%%%%%%%%%%%%%%%%%%%%%%%%
%%%%%%%%%%%%%%%%%%%%%%%%%%%%%%%%%%%%%%%%%%%%%%%%%%%%%%%%%%%%%%%%%%%%%%%%%%%%%%%
%%%%%%%%%%%%%%%%%%%%%%%%%%%%%%%%%%%%%%%%%%%%%%%%%%%%%%%%%%%%%%%%%%%%%%%%%%%%%%%
\begin{frame}[fragile]{\insertsection: la herramienta \bftt{make}}
  \begin{itemize}
    \item El archivo \bftt{Makefile} básico para compilar el documento
      con referencias cruzadas sería,
    \begin{center}
      \begin{minted}[fontsize=\tiny,frame=single]{makefile}
structure-crossref.pdf: structure-crossref.tex
	pdflatex structure-crossref
	pdflatex structure-crossref
      \end{minted}
    \end{center}
  \item Para ejecutar y realizar la compilación se debería lanzar el
    comando \bftt{make} en la consola (ubicado ya en el directorio
    donde se encuentran tanto el \bftt{Makefile} como también los
    archivos fuentes.
    \begin{center}
      \begin{minted}[fontsize=\tiny,frame=single]{rout}
luis@luis-laptop:$ ls
Makefile  structure-crossref.tex
luis@luis-laptop:$ make
pdflatex structure-crossref
This is pdfTeX, Version 3.1415926-2.4-1.40.13 (TeX Live 2012/Debian)
...
Output written on structure-crossref.pdf (1 page, 47567 bytes).
Transcript written on structure-crossref.log.
pdflatex structure-crossref
This is pdfTeX, Version 3.1415926-2.4-1.40.13 (TeX Live 2012/Debian)
...
Output written on structure-crossref.pdf (1 page, 38880 bytes).
Transcript written on structure-crossref.log.
      \end{minted}
    \end{center}
  \end{itemize}
\end{frame}

%%%%%%%%%%%%%%%%%%%%%%%%%%%%%%%%%%%%%%%%%%%%%%%%%%%%%%%%%%%%%%%%%%%%%%%%%%%%%%%
%%%%%%%%%%%%%%%%%%%%%%%%%%%%%%%%%%%%%%%%%%%%%%%%%%%%%%%%%%%%%%%%%%%%%%%%%%%%%%%
%%%%%%%%%%%%%%%%%%%%%%%%%%%%%%%%%%%%%%%%%%%%%%%%%%%%%%%%%%%%%%%%%%%%%%%%%%%%%%%
\begin{frame}[fragile]{\insertsection: la herramienta \bftt{make}}
  \begin{itemize}
    \item Se pueden programar mecanismos independientes y la
      herramienta procesará al ser llamados por su etiqueta,
    \begin{center}
      \begin{minted}[fontsize=\tiny,frame=single]{makefile}
structure-crossref.pdf: structure-crossref.tex
	pdflatex structure-crossref
	pdflatex structure-crossref

file-to-send.tgz:  structure-crossref.tex structure-crossref.pdf
	tar czf latex-course.tgz $^

clean:
	rm -f *.log *.aux 
      \end{minted}
    \end{center}
  \item En el caso de querer eliminar todos los archivos temporales
    podemos llamar a la función \bftt{clean} definida en el ítem
    anterior.
    \begin{center}
      \begin{minted}[fontsize=\tiny,frame=single]{rout}
luis@luis-laptop:$ ls
Makefile  structure-crossref.tex structure-crossref.log structure-crossref.aux
luis@luis-laptop:$ make clean
rm -f *.log *.aux 
luis@luis-laptop:$ ls
Makefile  structure-crossref.pdf  structure-crossref.tex
      \end{minted}
    \end{center}
  \end{itemize}
\end{frame}

%%%%%%%%%%%%%%%%%%%%%%%%%%%%%%%%%%%%%%%%%%%%%%%%%%%%%%%%%%%%%%%%%%%%%%%%%%%%%%% 
%%%%%%%%%%%%%%%%%%%%%%%%%%%%%%%%%%%%%%%%%%%%%%%%%%%%%%%%%%%%%%%%%%%%%%%%%%%%%%% 
%%%%%%%%%%%%%%%%%%%%%%%%%%%%%%%%%%%%%%%%%%%%%%%%%%%%%%%%%%%%%%%%%%%%%%%%%%%%%%%

\begin{frame}
  \begin{center}
    Thanks!
  \end{center}
\end{frame}

\end{document}
