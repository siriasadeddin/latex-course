\documentclass{beamer}
%\documentclass[draft]{beamer}

\input{preamble2.tex} %este preábulo es agregado por guanucoluis

\subtitle{Parte 4: Adaptando el documento a nuestras necesidades}

\begin{document}

%%%%%%%%%%%%%%%%%%%%%%%%%%%%%%%%%%%%%%%%%%%%%%%%%%%%%%%%%%%%%%%%%%%%%%%%%%%%%%%
%%%%%%%%%%%%%%%%%%%%%%%%%%%%%%%%%%%%%%%%%%%%%%%%%%%%%%%%%%%%%%%%%%%%%%%%%%%%%%%
%%%%%%%%%%%%%%%%%%%%%%%%%%%%%%%%%%%%%%%%%%%%%%%%%%%%%%%%%%%%%%%%%%%%%%%%%%%%%%%
\begin{frame}
  \titlepage
\end{frame}

%%%%%%%%%%%%%%%%%%%%%%%%%%%%%%%%%%%%%%%%%%%%%%%%%%%%%%%%%%%%%%%%%%%%%%%%%%%%%%%
%%%%%%%%%%%%%%%%%%%%%%%%%%%%%%%%%%%%%%%%%%%%%%%%%%%%%%%%%%%%%%%%%%%%%%%%%%%%%%%
%%%%%%%%%%%%%%%%%%%%%%%%%%%%%%%%%%%%%%%%%%%%%%%%%%%%%%%%%%%%%%%%%%%%%%%%%%%%%%%
\section{Estilo De Las Páginas}

%%%%%%%%%%%%%%%%%%%%%%%%%%%%%%%%%%%%%%%%%%%%%%%%%%%%%%%%%%%%%%%%%%%%%%%%%%%%%%%
%%%%%%%%%%%%%%%%%%%%%%%%%%%%%%%%%%%%%%%%%%%%%%%%%%%%%%%%%%%%%%%%%%%%%%%%%%%%%%%
%%%%%%%%%%%%%%%%%%%%%%%%%%%%%%%%%%%%%%%%%%%%%%%%%%%%%%%%%%%%%%%%%%%%%%%%%%%%%%%
\begin{frame}[fragile]{\insertsection: Nativos}
  \begin{itemize}
  \item \LaTeX{} soporta diferentes combinaciones de cabeceras y
    pies de páginas. \cmdbs{pagestyle} define cuál emplearse.
    \begin{itemize}
    \item \bftt{empty}
    \item \bftt{plain}
    \item \bftt{headings}      
    \item \bftt{myheadings}      
    \end{itemize}
  \item Es posible cambiar el estilo de la página actual con la orden
    pies de páginas \cmdbs{thispagestyle}.
  \end{itemize}
\end{frame}

%%%%%%%%%%%%%%%%%%%%%%%%%%%%%%%%%%%%%%%%%%%%%%%%%%%%%%%%%%%%%%%%%%%%%%%%%%%%%%%
%%%%%%%%%%%%%%%%%%%%%%%%%%%%%%%%%%%%%%%%%%%%%%%%%%%%%%%%%%%%%%%%%%%%%%%%%%%%%%%
%%%%%%%%%%%%%%%%%%%%%%%%%%%%%%%%%%%%%%%%%%%%%%%%%%%%%%%%%%%%%%%%%%%%%%%%%%%%%%% 
\begin{frame}[fragile]{\insertsection: Personalizados}
  El estilo \bftt{myheadings} permite modificar el contenido de
  la cabecera.\\[2ex]
  \begin{minipage}{0.5\linewidth}
    \inputminted[fontsize=\scriptsize,frame=single,resetmargins]{latex}%
    {pagestyle-example-myheadings.tex}
  \end{minipage}
  \begin{minipage}{0.4\linewidth}
    % trim: l b r t
    \includegraphics[width=\textwidth,clip,trim=1in 1.1in 1in 1.1in,page=6]{pagestyle-example-myheadings.pdf}
  \end{minipage}
\end{frame}

%%%%%%%%%%%%%%%%%%%%%%%%%%%%%%%%%%%%%%%%%%%%%%%%%%%%%%%%%%%%%%%%%%%%%%%%%%%%%%% 
%%%%%%%%%%%%%%%%%%%%%%%%%%%%%%%%%%%%%%%%%%%%%%%%%%%%%%%%%%%%%%%%%%%%%%%%%%%%%%%
%%%%%%%%%%%%%%%%%%%%%%%%%%%%%%%%%%%%%%%%%%%%%%%%%%%%%%%%%%%%%%%%%%%%%%%%%%%%%%%

\begin{frame}[fragile]{\insertsection: Personalizados}
  El paquete \bftt{fancyhdr} provee comandos para definir el
  contenido del lado izquierdo, centro y derecho, tanto del
  encabezado como el pie de página.\\[2ex]
  \begin{minipage}{0.5\linewidth}
    \inputminted[fontsize=\tiny,frame=single,resetmargins]{latex}%
    {pagestyle-example-fancyhdr.tex}
  \end{minipage}
  \begin{minipage}{0.4\linewidth}
    % trim: l b r t
    \includegraphics[width=\textwidth,clip,trim=1in 1.1in 1in 1.1in,page=6]{pagestyle-example-fancyhdr.pdf}
  \end{minipage}
\end{frame}

%%%%%%%%%%%%%%%%%%%%%%%%%%%%%%%%%%%%%%%%%%%%%%%%%%%%%%%%%%%%%%%%%%%%%%%%%%%%%%%
%%%%%%%%%%%%%%%%%%%%%%%%%%%%%%%%%%%%%%%%%%%%%%%%%%%%%%%%%%%%%%%%%%%%%%%%%%%%%%%
%%%%%%%%%%%%%%%%%%%%%%%%%%%%%%%%%%%%%%%%%%%%%%%%%%%%%%%%%%%%%%%%%%%%%%%%%%%%%%%

\section{Código Fuente en \LaTeX}

%%%%%%%%%%%%%%%%%%%%%%%%%%%%%%%%%%%%%%%%%%%%%%%%%%%%%%%%%%%%%%%%%%%%%%%%%%%%%%% 
%%%%%%%%%%%%%%%%%%%%%%%%%%%%%%%%%%%%%%%%%%%%%%%%%%%%%%%%%%%%%%%%%%%%%%%%%%%%%%% 
%%%%%%%%%%%%%%%%%%%%%%%%%%%%%%%%%%%%%%%%%%%%%%%%%%%%%%%%%%%%%%%%%%%%%%%%%%%%%%%


\begin{frame}[fragile]{\insertsection: Entorno \bftt{verbatim}}
  \begin{itemize}
  \item El texto encerrado entre \cmdbegin{verbatim} y
    \cmdend{verbatim} se escribirá directamente, con todos los saltos
    de línea y espacios, sin ejecutar ninguna orden \LaTeX.\\[1ex]
    \begin{exampletwouptiny}
\begin{verbatim}
#include<stdio.h>

int main()
{
  printf("Hello World");
  return 0;
}
\end{verbatim}
    \end{exampletwouptiny}
    \vskip 2ex
  \item Dentro de un párrafo, un comportamiento similar se puede 
    obtener con \cmdbs{verb}+text+.
  \end{itemize}
\end{frame}

%%%%%%%%%%%%%%%%%%%%%%%%%%%%%%%%%%%%%%%%%%%%%%%%%%%%%%%%%%%%%%%%%%%%%%%%%%%%%%% 
%%%%%%%%%%%%%%%%%%%%%%%%%%%%%%%%%%%%%%%%%%%%%%%%%%%%%%%%%%%%%%%%%%%%%%%%%%%%%%% 
%%%%%%%%%%%%%%%%%%%%%%%%%%%%%%%%%%%%%%%%%%%%%%%%%%%%%%%%%%%%%%%%%%%%%%%%%%%%%%%


\begin{frame}[fragile]{\insertsection: Paquetes \bftt{verbatim} y \bftt{fancyvrb}}
  \begin{itemize}
  \item El paquete \bftt{verbatim} nos permite incluir un fichero de
    texto como si estuviera dentro de un entorno \bftt{verbatim}.\\[2ex]
    \begin{exampletwouptiny}
\verbatiminput{main.c}
    \end{exampletwouptiny}
  \end{itemize}
\end{frame}

%%%%%%%%%%%%%%%%%%%%%%%%%%%%%%%%%%%%%%%%%%%%%%%%%%%%%%%%%%%%%%%%%%%%%%%%%%%%%%% 
%%%%%%%%%%%%%%%%%%%%%%%%%%%%%%%%%%%%%%%%%%%%%%%%%%%%%%%%%%%%%%%%%%%%%%%%%%%%%%% 
%%%%%%%%%%%%%%%%%%%%%%%%%%%%%%%%%%%%%%%%%%%%%%%%%%%%%%%%%%%%%%%%%%%%%%%%%%%%%%%


\begin{frame}[fragile]{\insertsection: Paquetes \bftt{verbatim} y \bftt{fancyvrb}}
  \begin{itemize}
  \item {\small El paquete \bftt{fancyvrb} nos permite incluir un fichero de
      texto como si estuviera dentro de un entorno \bftt{verbatim}.}\\[2ex]
  \item {\small Con el paquete \bftt{fancyvrb} se puede realizar
      tareas comunes a código-fuente, tales como: cambiar la fuente
      del texto y tamaño, numerar las líneas, etc.}\\[2ex]
    \begin{exampletwouptiny}
\VerbatimInput[frame=lines,
fontshape=sl,
fontsize=\scriptsize,
numbers=left,
formatcom=\color{blue}]
{main.c}
    \end{exampletwouptiny}

  \end{itemize}
\end{frame}

%%%%%%%%%%%%%%%%%%%%%%%%%%%%%%%%%%%%%%%%%%%%%%%%%%%%%%%%%%%%%%%%%%%%%%%%%%%%%%% 
%%%%%%%%%%%%%%%%%%%%%%%%%%%%%%%%%%%%%%%%%%%%%%%%%%%%%%%%%%%%%%%%%%%%%%%%%%%%%%% 
%%%%%%%%%%%%%%%%%%%%%%%%%%%%%%%%%%%%%%%%%%%%%%%%%%%%%%%%%%%%%%%%%%%%%%%%%%%%%%%


\begin{frame}[fragile]{\insertsection: Paquete \bftt{listings}}
  \begin{itemize}
  \item {\small El paquete \bftt{listings} se utiliza para imprimir
      código-fuente en \LaTeX{}. El entorno es similar al paquete 
      \bftt{verbatim}}.
    \begin{exampletwouptiny}
\begin{lstlisting}
  #include<stdio.h>
  
  int main()
  {
    printf("Hello World");
    return 0;
  }
\end{lstlisting}
    \end{exampletwouptiny}
    \vskip 2ex
  \item {\small Para personalizar el entorno \bftt{listings} se
      utiliza el comando \cmdbs{lstset}. El código siguiente debe ser
      agregado en el preámbulo del archivo \LaTeX{}.}
    \begin{minted}[fontsize=\scriptsize,frame=single]{latex}
\usepackage{listings}
\lstset{
  basicstyle=\tiny, language=c, fancyvrb=false, numbers=left,
  keywordstyle=\color{blue}\bfseries, frame=shadowbox,
  morekeywords={printf}, breaklines=true, 
  rulesepcolor=\color{blue}, stringstyle=\ttfamily}
    \end{minted}
  \end{itemize}
\end{frame}

%%%%%%%%%%%%%%%%%%%%%%%%%%%%%%%%%%%%%%%%%%%%%%%%%%%%%%%%%%%%%%%%%%%%%%%%%%%%%%% 
%%%%%%%%%%%%%%%%%%%%%%%%%%%%%%%%%%%%%%%%%%%%%%%%%%%%%%%%%%%%%%%%%%%%%%%%%%%%%%% 
%%%%%%%%%%%%%%%%%%%%%%%%%%%%%%%%%%%%%%%%%%%%%%%%%%%%%%%%%%%%%%%%%%%%%%%%%%%%%%%


\begin{frame}[fragile]{\insertsection: Paquete \bftt{minted}}
  \begin{itemize}
  \item {\small El paquete \bftt{minted} permite insertar código
      fuente y resaltar las la sintaxis del lenguaje utilizado. Este
      paquete utiliza una librería del lenguaje Python
      (\bftt{python-pygments}).}
    \vskip 2ex
    \begin{exampletwouptiny}
\begin{minted}[fontsize=\tiny,
  frame=single,
  linenos=true]{c}
  #include<stdio.h>
  
  int main()
  {
    printf("Hello World");
    return 0;
  }
\end{minted}
    \end{exampletwouptiny}
    \vskip 2ex
  \item {\small Los últimos dos paquetes, \bftt{listings} y
      \bftt{minted}, requieren otros paquetes \LaTeX{} (y en el caso de
      \bftt{minted} paquetes externos) para poder funcionar
      correctamente. Se aconseja leer las respectivas documentaciones de
      la página \href{}{http://ctan.org/}}
  \end{itemize}
\end{frame}

%%%%%%%%%%%%%%%%%%%%%%%%%%%%%%%%%%%%%%%%%%%%%%%%%%%%%%%%%%%%%%%%%%%%%%%%%%%%%%%
%%%%%%%%%%%%%%%%%%%%%%%%%%%%%%%%%%%%%%%%%%%%%%%%%%%%%%%%%%%%%%%%%%%%%%%%%%%%%%%
%%%%%%%%%%%%%%%%%%%%%%%%%%%%%%%%%%%%%%%%%%%%%%%%%%%%%%%%%%%%%%%%%%%%%%%%%%%%%%%

\section{Figuras (continuación)}


%%%%%%%%%%%%%%%%%%%%%%%%%%%%%%%%%%%%%%%%%%%%%%%%%%%%%%%%%%%%%%%%%%%%%%%%%%%%%%%
%%%%%%%%%%%%%%%%%%%%%%%%%%%%%%%%%%%%%%%%%%%%%%%%%%%%%%%%%%%%%%%%%%%%%%%%%%%%%%%
%%%%%%%%%%%%%%%%%%%%%%%%%%%%%%%%%%%%%%%%%%%%%%%%%%%%%%%%%%%%%%%%%%%%%%%%%%%%%%%
\begin{frame}[fragile]{\insertsection: Subfiguras}
  \begin{itemize}
  \item Anteriormente
    (\href{https://raw.github.com/guanucoluis/latex-course/master/es/part2.pdf}{Parte
      2}) vimos cómo insertar imágenes en \LaTeX{}. En esta parte trataremos
    casos específicos de la inserción de figuras.
  \item En el siguiente ejemplo tenemos dos imágenes que se encuentran
    vinculadas entre sí.
    \begin{minipage}{0.5\linewidth}
      \inputminted[fontsize=\tiny,frame=single,resetmargins]{latex}%
      {multiple-figures.tex}
    \end{minipage}
    \begin{minipage}{0.4\linewidth}
      % trim: l b r t
      \includegraphics[width=\textwidth,clip,trim=1in 1.1in 1in 1.1in,page=4]{multiple-figures.pdf}   
    \end{minipage}
    \end{itemize}
\end{frame}

%%%%%%%%%%%%%%%%%%%%%%%%%%%%%%%%%%%%%%%%%%%%%%%%%%%%%%%%%%%%%%%%%%%%%%%%%%%%%%%
%%%%%%%%%%%%%%%%%%%%%%%%%%%%%%%%%%%%%%%%%%%%%%%%%%%%%%%%%%%%%%%%%%%%%%%%%%%%%%%
%%%%%%%%%%%%%%%%%%%%%%%%%%%%%%%%%%%%%%%%%%%%%%%%%%%%%%%%%%%%%%%%%%%%%%%%%%%%%%%
\begin{frame}[fragile]{\insertsection: Subfiguras}
  \begin{itemize}
  \item Para mejorar la hoja anterior se usan los paquetes
    \bftt{caption} y \bftt{subcaption}. De esta forma se pueden agregar
    sub-flotantes en un único flotante.\\
    \begin{minipage}{0.5\linewidth}
      \inputminted[fontsize=\tiny,frame=single,resetmargins]{latex}%
      {subfigure-example.tex}
    \end{minipage}
    \begin{minipage}{0.4\linewidth}
      % trim: l b r t
      \includegraphics[width=\textwidth,clip,trim=1in 1.1in 1in 1.1in,page=4]{subfigure-example.pdf}   
    \end{minipage}
    \end{itemize}
\end{frame}

%%%%%%%%%%%%%%%%%%%%%%%%%%%%%%%%%%%%%%%%%%%%%%%%%%%%%%%%%%%%%%%%%%%%%%%%%%%%%%% 
%%%%%%%%%%%%%%%%%%%%%%%%%%%%%%%%%%%%%%%%%%%%%%%%%%%%%%%%%%%%%%%%%%%%%%%%%%%%%%%
%%%%%%%%%%%%%%%%%%%%%%%%%%%%%%%%%%%%%%%%%%%%%%%%%%%%%%%%%%%%%%%%%%%%%%%%%%%%%%%
\section{Bibliografía (continuación)}

%%%%%%%%%%%%%%%%%%%%%%%%%%%%%%%%%%%%%%%%%%%%%%%%%%%%%%%%%%%%%%%%%%%%%%%%%%%%%%%
%%%%%%%%%%%%%%%%%%%%%%%%%%%%%%%%%%%%%%%%%%%%%%%%%%%%%%%%%%%%%%%%%%%%%%%%%%%%%%%
%%%%%%%%%%%%%%%%%%%%%%%%%%%%%%%%%%%%%%%%%%%%%%%%%%%%%%%%%%%%%%%%%%%%%%%%%%%%%%%
\begin{frame}[fragile]
  \frametitle{\insertsection: El entorno \bftt{thebibliography}}
  \begin{itemize}
  \item En la
    \href{https://raw.github.com/guanucoluis/latex-course/master/es/part2.pdf}{Parte
      2} se mostró como utilizar bases de datos de bibliografías para
    nuestros documentos \LaTeX{}. Pero para el caso de querer generar
    un simple reporte, el proceso de compilación con \bftt{bibtex}
    resulta lento.
  \item \LaTeX{} provee un entorno llamado \bftt{thebibliography}. De
    esta forma se puede agregar bibliografía en nuestro documento sin
    la necesidad de llamar a \bftt{bibtex}.
    \vskip 2ex
    \begin{minted}[fontsize=\small,frame=single]{latex}
\begin{thebibliography}{1}

\bibitem{lamport94}
  Leslie Lamport,
  \emph{\LaTeX: a document preparation system},
  Addison Wesley, Massachusetts,
  2nd edition,
  1994.

\end{thebibliography}
    \end{minted}
  \end{itemize}
\end{frame}

%%%%%%%%%%%%%%%%%%%%%%%%%%%%%%%%%%%%%%%%%%%%%%%%%%%%%%%%%%%%%%%%%%%%%%%%%%%%%%%
%%%%%%%%%%%%%%%%%%%%%%%%%%%%%%%%%%%%%%%%%%%%%%%%%%%%%%%%%%%%%%%%%%%%%%%%%%%%%%%
%%%%%%%%%%%%%%%%%%%%%%%%%%%%%%%%%%%%%%%%%%%%%%%%%%%%%%%%%%%%%%%%%%%%%%%%%%%%%%%
\begin{frame}[fragile]{\insertsection: El entorno \bftt{thebibliography}}
  \begin{itemize}
  \item A continuación se muestra el mismo ejemplo utilizado con
    \bftt{bibtex}.
    \vskip 2ex
    \begin{minipage}{0.5\linewidth}
      \inputminted[fontsize=\tiny,frame=single,resetmargins]{latex}%
      {biblio-example.tex}
    \end{minipage}
    \begin{minipage}{0.4\linewidth}
      % trim: l b r t
      \includegraphics[width=\textwidth,clip,trim=1in 1.1in 1in 1.1in]{biblio-example.pdf}   
    \end{minipage}
    \end{itemize}
\end{frame}

%%%%%%%%%%%%%%%%%%%%%%%%%%%%%%%%%%%%%%%%%%%%%%%%%%%%%%%%%%%%%%%%%%%%%%%%%%%%%%
%%%%%%%%%%%%%%%%%%%%%%%%%%%%%%%%%%%%%%%%%%%%%%%%%%%%%%%%%%%%%%%%%%%%%%%%%%%%%%
%%%%%%%%%%%%%%%%%%%%%%%%%%%%%%%%%%%%%%%%%%%%%%%%%%%%%%%%%%%%%%%%%%%%%%%%%%%%%%
\section{Dibujando con circui\protect\tikzname}

%%%%%%%%%%%%%%%%%%%%%%%%%%%%%%%%%%%%%%%%%%%%%%%%%%%%%%%%%%%%%%%%%%%%%%%%%%%%%%%
%%%%%%%%%%%%%%%%%%%%%%%%%%%%%%%%%%%%%%%%%%%%%%%%%%%%%%%%%%%%%%%%%%%%%%%%%%%%%%%
%%%%%%%%%%%%%%%%%%%%%%%%%%%%%%%%%%%%%%%%%%%%%%%%%%%%%%%%%%%%%%%%%%%%%%%%%%%%%%%
\begin{frame}[fragile]{\insertsection}
  \begin{itemize}
    \item El paquete circui\tikzname{} provee macros para componer
      diagramas eléctricos/electrónicos  en \LaTeX{}.
    \item Solo debemos cargar el paquete \bftt{circuitikz}. Éste
      cargará automáticamente el paquete \bftt{tikz}.
    \vskip 2ex
    \begin{exampletwouptiny}
\ctikzset{bipoles/length=1cm}
\begin{circuitikz}
  [scale=0.9]
  \draw[help lines, green]
  (0,0) grid (2,2);
  \draw 
  (0,0) to[sV=5<\volt>]
  (0,2) to[R=47<\ohm>]
  (2,2) to[L=5<\milli\henry>]
  (2,0) to[C=10<\micro\farad>]
  (0,0);
\end{circuitikz}
    \end{exampletwouptiny}
  \end{itemize}
\end{frame}

%%%%%%%%%%%%%%%%%%%%%%%%%%%%%%%%%%%%%%%%%%%%%%%%%%%%%%%%%%%%%%%%%%%%%%%%%%%%%%%
%%%%%%%%%%%%%%%%%%%%%%%%%%%%%%%%%%%%%%%%%%%%%%%%%%%%%%%%%%%%%%%%%%%%%%%%%%%%%%%
%%%%%%%%%%%%%%%%%%%%%%%%%%%%%%%%%%%%%%%%%%%%%%%%%%%%%%%%%%%%%%%%%%%%%%%%%%%%%%%
\begin{frame}[fragile]{\insertsection}
  \begin{itemize}
  \item Al igual que muchos paquetes que hemos utilizado,
    circui\tikzname{} permite recibir opciones para personalizar su uso.
    \vskip 2ex
    \begin{minted}[fontsize=\footnotesize,frame=single]{latex}
\usepackage[american,cuteinductors,siunitx]{circuitikz}
\usepackage{siunitx}
    \end{minted}
  \item En función de los argumentos opcionales del ejemplo anterior tenemos:
    \begin{description}
    \item[\bftt{american}] define qué simbología se utilizará. En este
      caso las estándares americanas.
    \item[\bftt{cuteinductors}] permite que el símbolo del inductor
      se encuentre más ondulado que el estándar.
    \item[\bftt{siunitx}] informa a \tikzname{} que utilizará las
      definiciones de unidades desde el paquete \bftt{siunitx}.
    \end{description}
  \end{itemize}
\end{frame}

%%%%%%%%%%%%%%%%%%%%%%%%%%%%%%%%%%%%%%%%%%%%%%%%%%%%%%%%%%%%%%%%%%%%%%%%%%%%%%%
%%%%%%%%%%%%%%%%%%%%%%%%%%%%%%%%%%%%%%%%%%%%%%%%%%%%%%%%%%%%%%%%%%%%%%%%%%%%%%%
%%%%%%%%%%%%%%%%%%%%%%%%%%%%%%%%%%%%%%%%%%%%%%%%%%%%%%%%%%%%%%%%%%%%%%%%%%%%%%%
\begin{frame}[fragile]{\insertsection: Tipos de Componentes}
  \begin{itemize}
  \item Monopolares\\
    \begin{exampletwouptiny}
\begin{circuitikz}
  \draw node[ground]{GND};
\end{circuitikz}
    \end{exampletwouptiny}
  \item Bipolares\\
    \begin{exampletwouptiny}
\begin{circuitikz} 
\ctikzset{bipoles/length=1cm}
\draw (0,0) to
[R=$R_1$,i=$i_1$,v=$v_1$,o-*] (2,0);
\end{circuitikz}
    \end{exampletwouptiny}
  \item Tripolar\\
    \begin{exampletwouptiny}
\begin{circuitikz}
  \draw (0,0) node[npn](npn){T1}
  (npn.base) node[anchor=east]{B}
  (npn.collector) node[anchor=south]{C}
  (npn.emitter) node[anchor=north]{E};
\end{circuitikz}
    \end{exampletwouptiny}
  \end{itemize}
\end{frame}


%%%%%%%%%%%%%%%%%%%%%%%%%%%%%%%%%%%%%%%%%%%%%%%%%%%%%%%%%%%%%%%%%%%%%%%%%%%%%%%
%%%%%%%%%%%%%%%%%%%%%%%%%%%%%%%%%%%%%%%%%%%%%%%%%%%%%%%%%%%%%%%%%%%%%%%%%%%%%%%
%%%%%%%%%%%%%%%%%%%%%%%%%%%%%%%%%%%%%%%%%%%%%%%%%%%%%%%%%%%%%%%%%%%%%%%%%%%%%%%
\begin{frame}[fragile]{\insertsection: Tipos de Componentes}
  \begin{itemize}
  \item Doble-bipolares\\[1ex]
    \begin{exampletwouptiny}
\begin{circuitikz} \draw 
  (0,0) node[transformer](T){}
  (T.A1) node[anchor=east] {A1}
  (T.A2) node[anchor=east] {A2}
  (T.B1) node[anchor=west] {B1}
  (T.B2) node[anchor=west] {B2}
  (T.base) node{N=$\frac{1}{20}$};
\end{circuitikz}
    \end{exampletwouptiny}
  \item Compuertas lógicas\\[1ex]
    \begin{exampletwouptiny}
\begin{circuitikz} \draw 
  (0,0) node[and port](myand) {}
  (myand.in 1) node[anchor=east] {A}
  (myand.in 2) node[anchor=east] {B}
  (myand.out) node[anchor=west] {O};
  \draw (myand) 
  node[below=6mm]{$O = A \cdot B$};
\end{circuitikz}
    \end{exampletwouptiny}
  \end{itemize}
\end{frame}


%%%%%%%%%%%%%%%%%%%%%%%%%%%%%%%%%%%%%%%%%%%%%%%%%%%%%%%%%%%%%%%%%%%%%%%%%%%%%%%
%%%%%%%%%%%%%%%%%%%%%%%%%%%%%%%%%%%%%%%%%%%%%%%%%%%%%%%%%%%%%%%%%%%%%%%%%%%%%%%
%%%%%%%%%%%%%%%%%%%%%%%%%%%%%%%%%%%%%%%%%%%%%%%%%%%%%%%%%%%%%%%%%%%%%%%%%%%%%%%
\begin{frame}[fragile]{\insertsection: Tipos de Componentes}
  \begin{itemize}
  \item Amplificadores
    \begin{itemize}
    \item Operacionales\\[1ex]
    \begin{exampletwouptiny}
\begin{circuitikz} \draw 
  (0,0) node[op amp] (opamp){}
  (opamp.+) node[left]{$v_+$}
  (opamp.-) node[left]{$v_-$}
  (opamp.out) node[right]{$v_o$}
  (opamp.down) node[ground]{}
  (opamp.up) ++ (0,.5) node[above]
  {\SI{12}{\volt}} -- (opamp.up);
\end{circuitikz}
    \end{exampletwouptiny}
    \item \textsl{Buffers}\\[1ex]
    \begin{exampletwouptiny}
\begin{circuitikz} \draw 
  (0,0) node[buffer] (buf){}
  (buf.in) node[left]{In}
  (buf.out) node[right]{Out};
\end{circuitikz}
    \end{exampletwouptiny}
    \end{itemize}
  \end{itemize}
\end{frame}

%%%%%%%%%%%%%%%%%%%%%%%%%%%%%%%%%%%%%%%%%%%%%%%%%%%%%%%%%%%%%%%%%%%%%%%%%%%%%%%
%%%%%%%%%%%%%%%%%%%%%%%%%%%%%%%%%%%%%%%%%%%%%%%%%%%%%%%%%%%%%%%%%%%%%%%%%%%%%%%
%%%%%%%%%%%%%%%%%%%%%%%%%%%%%%%%%%%%%%%%%%%%%%%%%%%%%%%%%%%%%%%%%%%%%%%%%%%%%%%
\begin{frame}[fragile]{\insertsection: Integración con \bftt{siunitx}}
  \begin{itemize}
  \item Existen dos modos de utilizar las unidades con el paquete
    \bftt{siunitx}\\[1ex]
    \begin{exampletwouptiny}
\begin{circuitikz} \draw 
  (0,0) to[R, l=1<\kilo\ohm>]
  (2,0);
\end{circuitikz}
    \end{exampletwouptiny}
    \vskip 2ex
    \begin{exampletwouptiny}
\begin{circuitikz} \draw 
  (0,0) to[R, l=\SI{1}{\kilo\ohm}]
  (2,0);
\end{circuitikz}
    \end{exampletwouptiny}
  \end{itemize}
\end{frame}


%%%%%%%%%%%%%%%%%%%%%%%%%%%%%%%%%%%%%%%%%%%%%%%%%%%%%%%%%%%%%%%%%%%%%%%%%%%%%%%
%%%%%%%%%%%%%%%%%%%%%%%%%%%%%%%%%%%%%%%%%%%%%%%%%%%%%%%%%%%%%%%%%%%%%%%%%%%%%%%
%%%%%%%%%%%%%%%%%%%%%%%%%%%%%%%%%%%%%%%%%%%%%%%%%%%%%%%%%%%%%%%%%%%%%%%%%%%%%%%
\begin{frame}[fragile]{\insertsection: Espejado de componentes}
  \begin{itemize}
  \item En el momento de instanaciar un componente, se puede
    especificar que el símbolo sea espejado.\\[1ex]
    \begin{exampletwouptiny}
\begin{circuitikz} \draw 
  (0,0) to[ospst=T,v=v,i=$i_1$]
  (2,0);
\end{circuitikz}
    \end{exampletwouptiny}
    \vskip 2ex
    \begin{exampletwouptiny}
\begin{circuitikz} \draw 
  (0,0) to[ospst=T,mirror,v=v,i=$i_1$]
  (2,0);
\end{circuitikz}
    \end{exampletwouptiny}
  \item Tener en cuenta que el \bftt{label} también se modifica. Pero
    esto no sucede en el caso de la indicación de corriente.
    \vskip 2ex
    \begin{exampletwouptiny}
\begin{circuitikz} \draw 
  (0,0) to[ospst=T,mirror,v=v,i=$i_1$]
  (2,0);
\end{circuitikz}
    \end{exampletwouptiny}
  \end{itemize}
\end{frame}

%%%%%%%%%%%%%%%%%%%%%%%%%%%%%%%%%%%%%%%%%%%%%%%%%%%%%%%%%%%%%%%%%%%%%%%%%%%%%%%
%%%%%%%%%%%%%%%%%%%%%%%%%%%%%%%%%%%%%%%%%%%%%%%%%%%%%%%%%%%%%%%%%%%%%%%%%%%%%%%
%%%%%%%%%%%%%%%%%%%%%%%%%%%%%%%%%%%%%%%%%%%%%%%%%%%%%%%%%%%%%%%%%%%%%%%%%%%%%%%
\begin{frame}[fragile]{\insertsection: Parámetros de Circui\tikzname{}}
  \begin{itemize}
  \item La mayoría de los macros que provee circui\tikzname{} hacen
    uso de los \bftt{pgfkeys} para la configuración de las imágenes
    que ofrece. Recuerde que Circui\tikzname{} utiliza el paquete
    \tikzname{}.
  \item Para manipular las macros utilizamos el comando \cmdbs{ctikzset}.\\[1ex]
    \begin{exampletwouptiny}
\tikz \draw (0,0) to[R=1<\ohm>](2,0); 
\par
\ctikzset{bipoles/resistor/height=.6}
\tikz \draw (0,0) to[R=1<\ohm>](2,0);
    \end{exampletwouptiny}
    \vskip 2ex
    \begin{exampletwouptiny}
\tikz \draw 
(0,0) to[C, i=1<\milli\ampere>] (2,0); 
\par
\ctikzset{current/distance=.8}
\tikz \draw 
(0,0) to[C, i=1<\milli\ampere>] (2,0); 
    \end{exampletwouptiny}
  \end{itemize}
\end{frame}

%%%%%%%%%%%%%%%%%%%%%%%%%%%%%%%%%%%%%%%%%%%%%%%%%%%%%%%%%%%%%%%%%%%%%%%%%%%%%%%
%%%%%%%%%%%%%%%%%%%%%%%%%%%%%%%%%%%%%%%%%%%%%%%%%%%%%%%%%%%%%%%%%%%%%%%%%%%%%%%
%%%%%%%%%%%%%%%%%%%%%%%%%%%%%%%%%%%%%%%%%%%%%%%%%%%%%%%%%%%%%%%%%%%%%%%%%%%%%%%
\begin{frame}[fragile]{\insertsection: Tamaño de los componentes}
  \begin{itemize}
  \item Tal vez el parámetro más importante sea
    \cmdbs{circuitikzbasekey/bipoles/length}, que puede considerarse
    como el largo de un resistor; todas las demás longitudes serán
    relativa a este \bftt{key}.\\[1ex]
    \begin{minted}[fontsize=\footnotesize,frame=single]{latex}
\ctikzset{bipoles/length=1.4cm}
\begin{circuitikz}[scale=1.2]\draw
...
    \end{minted}
    \vskip 2ex
\ctikzset{bipoles/length=1.4cm}
\begin{center}
\begin{circuitikz}[scale=1.2]\draw
  (0,0) node[anchor=east] {B}
  to[short , o-*] (1,0)
  to[R=20<\ohm>, *-*] (1,2)
  to[R=10<\ohm>, v=$v_x$] (3,2) -- (4,2)
  to[ cI=$\frac{\si{\siemens}}{5} v_x$, *-*] (4,0) -- (3,0)
  to[R=5<\ohm>, *-*] (3,2)
  (3,0) -- (1,0)
  (1,2) to[short , -o] (0,2) node[anchor=east]{A};
\end{circuitikz}
\end{center}
  \end{itemize}
\end{frame}

%%%%%%%%%%%%%%%%%%%%%%%%%%%%%%%%%%%%%%%%%%%%%%%%%%%%%%%%%%%%%%%%%%%%%%%%%%%%%%%
%%%%%%%%%%%%%%%%%%%%%%%%%%%%%%%%%%%%%%%%%%%%%%%%%%%%%%%%%%%%%%%%%%%%%%%%%%%%%%%
%%%%%%%%%%%%%%%%%%%%%%%%%%%%%%%%%%%%%%%%%%%%%%%%%%%%%%%%%%%%%%%%%%%%%%%%%%%%%%%
\begin{frame}[fragile]{\insertsection: Tamaño de los componentes}
  \begin{itemize}
  \item Tal vez el parámetro más importante sea
    \cmdbs{circuitikzbasekey/bipoles/length}, que puede considerarse
    como el largo de un resistor; todas las demás longitudes serán
    relativa a este \bftt{key}.\\[1ex]
    \begin{minted}[fontsize=\footnotesize,frame=single]{latex}
\ctikzset{bipoles/length=.8cm}
\begin{circuitikz}[scale=1.2]\draw
...
    \end{minted}
    \vskip 2ex
\ctikzset{bipoles/length=.8cm}
\begin{center}
\begin{circuitikz}[scale=1.2]\draw
  (0,0) node[anchor=east] {B}
  to[short , o-*] (1,0)
  to[R=20<\ohm>, *-*] (1,2)
  to[R=10<\ohm>, v=$v_x$] (3,2) -- (4,2)
  to[ cI=$\frac{\si{\siemens}}{5} v_x$, *-*] (4,0) -- (3,0)
  to[R=5<\ohm>, *-*] (3,2)
  (3,0) -- (1,0)
  (1,2) to[short , -o] (0,2) node[anchor=east]{A};
\end{circuitikz}
\end{center}
  \end{itemize}
\end{frame}


%%%%%%%%%%%%%%%%%%%%%%%%%%%%%%%%%%%%%%%%%%%%%%%%%%%%%%%%%%%%%%%%%%%%%%%%%%%%%%%
%%%%%%%%%%%%%%%%%%%%%%%%%%%%%%%%%%%%%%%%%%%%%%%%%%%%%%%%%%%%%%%%%%%%%%%%%%%%%%%
%%%%%%%%%%%%%%%%%%%%%%%%%%%%%%%%%%%%%%%%%%%%%%%%%%%%%%%%%%%%%%%%%%%%%%%%%%%%%%%
\begin{frame}[fragile]{\insertsection: Varios ejemplos}
  \begin{itemize}
  \item Revise
    \href{http://texample.net}{\TeX{}ample.net} para
    muchos ejemplos de \fbox{\href{http://texample.net/tikz/examples/tag/circuitikz/}{Circui\tikzname{}}}:
  \end{itemize}
  \begin{figure}
    \href{http://texample.net/tikz/examples/mosfet/}{%
      \includegraphics[width=0.3\textwidth]{mosfet}}
    \href{http://texample.net/tikz/examples/collpits/}{%
      \includegraphics[width=0.3\textwidth]{collpits}}
    \href{http://texample.net/tikz/examples/4-bit-counter/}{%
      \includegraphics[width=0.3\textwidth]{4-bit-counter}}
  \end{figure}
\end{frame}

%%%%%%%%%%%%%%%%%%%%%%%%%%%%%%%%%%%%%%%%%%%%%%%%%%%%%%%%%%%%%%%%%%%%%%%%%%%%%%
%%%%%%%%%%%%%%%%%%%%%%%%%%%%%%%%%%%%%%%%%%%%%%%%%%%%%%%%%%%%%%%%%%%%%%%%%%%%%%
%%%%%%%%%%%%%%%%%%%%%%%%%%%%%%%%%%%%%%%%%%%%%%%%%%%%%%%%%%%%%%%%%%%%%%%%%%%%%%
\section{Compilación en Consola}

%%%%%%%%%%%%%%%%%%%%%%%%%%%%%%%%%%%%%%%%%%%%%%%%%%%%%%%%%%%%%%%%%%%%%%%%%%%%%%%
%%%%%%%%%%%%%%%%%%%%%%%%%%%%%%%%%%%%%%%%%%%%%%%%%%%%%%%%%%%%%%%%%%%%%%%%%%%%%%%
%%%%%%%%%%%%%%%%%%%%%%%%%%%%%%%%%%%%%%%%%%%%%%%%%%%%%%%%%%%%%%%%%%%%%%%%%%%%%%%
\begin{frame}[fragile]{\insertsection: pdf\LaTeX{}}
  \begin{itemize}
  \item Tal vez el parámetro más importante sea
    \cmdbs{circuitikzbasekey/bipoles/length}, que puede considerarse
    como el largo de un resistor; todas las demás longitudes serán
    relativa a este \bftt{key}.\\[1ex]
    \begin{minted}[fontsize=\footnotesize,frame=single]{latex}
\ctikzset{bipoles/length=.8cm}
\begin{circuitikz}[scale=1.2]\draw
...
    \end{minted}
    \vskip 2ex
\ctikzset{bipoles/length=.8cm}
\begin{center}
\begin{circuitikz}[scale=1.2]\draw
  (0,0) node[anchor=east] {B}
  to[short , o-*] (1,0)
  to[R=20<\ohm>, *-*] (1,2)
  to[R=10<\ohm>, v=$v_x$] (3,2) -- (4,2)
  to[ cI=$\frac{\si{\siemens}}{5} v_x$, *-*] (4,0) -- (3,0)
  to[R=5<\ohm>, *-*] (3,2)
  (3,0) -- (1,0)
  (1,2) to[short , -o] (0,2) node[anchor=east]{A};
\end{circuitikz}
\end{center}
  \end{itemize}
\end{frame}

%%%%%%%%%%%%%%%%%%%%%%%%%%%%%%%%%%%%%%%%%%%%%%%%%%%%%%%%%%%%%%%%%%%%%%%%%%%%%%% 
%%%%%%%%%%%%%%%%%%%%%%%%%%%%%%%%%%%%%%%%%%%%%%%%%%%%%%%%%%%%%%%%%%%%%%%%%%%%%%% 
%%%%%%%%%%%%%%%%%%%%%%%%%%%%%%%%%%%%%%%%%%%%%%%%%%%%%%%%%%%%%%%%%%%%%%%%%%%%%%%

\begin{frame}
  \begin{center}
    Thanks!
  \end{center}
\end{frame}

\end{document}
