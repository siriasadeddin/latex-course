\documentclass{beamer}

%
% Common preamble for all three parts.
%
\usepackage[spanish]{babel}
\usepackage{amsmath}
\usepackage{color}
\ProvidesPackage{minted}
\usepackage{minted}
%\setminted{encoding=utf8}
\usepackage{hyperref}
\usepackage{multicol}
\usepackage{tabularx}
\usepackage{tikz}

\usepackage[utf8]{inputenc}
%\usepackage{ucs}
%\usepackage[T1]{fontenc}
%\newcommand{\minted@encoding}{\minted@get@opt{encoding}{UTF8}}

% only inline todonotes work
\usepackage{xkeyval}
\usepackage[textsize=small]{todonotes}
\presetkeys{todonotes}{inline}{}

\usetikzlibrary{shapes,arrows,positioning,shadows}

% no nav buttons
\usenavigationsymbolstemplate{}

\newcommand{\bftt}[1]{\textbf{\texttt{#1}}}
\newcommand{\comment}[1]{{\color[HTML]{008080}\textit{\textbf{\texttt{#1}}}}}
\newcommand{\cmd}[1]{{\color[HTML]{008000}\bftt{#1}}}
\newcommand{\bs}{\char`\\}
\newcommand{\cmdbs}[1]{\cmd{\bs#1}}
\newcommand{\lcb}{\char '173}
\newcommand{\rcb}{\char '175}
\newcommand{\cmdbegin}[1]{\cmdbs{begin\lcb}\bftt{#1}\cmd{\rcb}}
\newcommand{\cmdend}[1]{\cmdbs{end\lcb}\bftt{#1}\cmd{\rcb}}

\newcommand{\wllogo}{\textbf{Overleaf}}

% this is where the example source files are loaded from
% do not include a trailing slash
\newcommand{\fileuri}{https://raw.github.com/guanucoluis/latex-course/master/es}

\newcommand{\wlserver}{https://www.overleaf.com}
\newcommand{\wlnewdoc}[1]{\wlserver/docs?snip\_uri=\fileuri/#1\&splash=none}

\def\tikzname{Ti\emph{k}Z}

% from http://tex.stackexchange.com/questions/5226/keyboard-font-for-latex
\newcommand*\keystroke[1]{%
  \tikz[baseline=(key.base)]
    \node[%
      draw,
      fill=white,
      drop shadow={shadow xshift=0.25ex,shadow yshift=-0.25ex,fill=black,opacity=0.75},
      rectangle,
      rounded corners=2pt,
      inner sep=1pt,
      line width=0.5pt,
      font=\scriptsize\sffamily
    ](key) {#1\strut}
  ;
}
\newcommand{\keystrokebftt}[1]{\keystroke{\bftt{#1}}}

% stolen from minted.dtx
\newenvironment{exampletwoup}
  {\VerbatimEnvironment
   \begin{VerbatimOut}{example.out}}
  {\end{VerbatimOut}
   \setlength{\parindent}{0pt}
   \fbox{\begin{tabular}{l|l}
   \begin{minipage}{0.55\linewidth}
     \inputminted[fontsize=\small,resetmargins]{latex}{example.out}
   \end{minipage} &
   \begin{minipage}{0.35\linewidth}
     \input{example.out}
   \end{minipage}
   \end{tabular}}}

\newenvironment{exampletwouptiny}
  {\VerbatimEnvironment
   \begin{VerbatimOut}{example.out}}
  {\end{VerbatimOut}
   \setlength{\parindent}{0pt}
   \fbox{\begin{tabular}{l|l}
   \begin{minipage}{0.55\linewidth}
     \inputminted[fontsize=\scriptsize,resetmargins]{latex}{example.out}
   \end{minipage} &
   \begin{minipage}{0.35\linewidth}
     \setlength{\parskip}{6pt plus 1pt minus 1pt}%
     \raggedright\scriptsize\input{example.out}
   \end{minipage}
   \end{tabular}}}

\newenvironment{exampletwouptinynoframe}
  {\VerbatimEnvironment
   \begin{VerbatimOut}{example.out}}
  {\end{VerbatimOut}
   \setlength{\parindent}{0pt}
   \begin{tabular}{l|l}
   \begin{minipage}{0.55\linewidth}
     \inputminted[fontsize=\scriptsize,resetmargins]{latex}{example.out}
   \end{minipage} &
   \begin{minipage}{0.35\linewidth}
     \setlength{\parskip}{6pt plus 1pt minus 1pt}%
     \raggedright\scriptsize\input{example.out}
   \end{minipage}
   \end{tabular}}

\title{Una Introducción Interactiva a \LaTeX}
\author{Siria Sadeddin}


\subtitle{Parte 1: Conceptos Básicos}

\begin{document}

%%%%%%%%%%%%%%%%%%%%%%%%%%%%%%%%%%%%%%%%%%%%%%%%%%%%%%%%%%%%%%%%%%%%%%%%%%%%%%% 
%%%%%%%%%%%%%%%%%%%%%%%%%%%%%%%%%%%%%%%%%%%%%%%%%%%%%%%%%%%%%%%%%%%%%%%%%%%%%%% 
%%%%%%%%%%%%%%%%%%%%%%%%%%%%%%%%%%%%%%%%%%%%%%%%%%%%%%%%%%%%%%%%%%%%%%%%%%%%%%%
\begin{frame}
  \titlepage
\end{frame}

%%%%%%%%%%%%%%%%%%%%%%%%%%%%%%%%%%%%%%%%%%%%%%%%%%%%%%%%%%%%%%%%%%%%%%%%%%%%%%%
%%%%%%%%%%%%%%%%%%%%%%%%%%%%%%%%%%%%%%%%%%%%%%%%%%%%%%%%%%%%%%%%%%%%%%%%%%%%%%%
%%%%%%%%%%%%%%%%%%%%%%%%%%%%%%%%%%%%%%%%%%%%%%%%%%%%%%%%%%%%%%%%%%%%%%%%%%%%%%%
\begin{frame}{¿Por qué \LaTeX{}?}
  \begin{itemize}
  \item Logra magníficos documentos
    \begin{itemize}
    \item Especialmente los matemáticos
    \end{itemize}
    % 
  \item Fue creado por científicos, para científicos
    \begin{itemize}
    \item Una amplia y activa  comunidad
    \end{itemize}
    % 
  \item Es de gran alcance --- puede extenderlo
    \begin{itemize}
    \item Paquetes para publicaciones científicas, presentaciones, hojas de cálculos, \ldots
    \end{itemize}
  \end{itemize}
\end{frame}

%%%%%%%%%%%%%%%%%%%%%%%%%%%%%%%%%%%%%%%%%%%%%%%%%%%%%%%%%%%%%%%%%%%%%%%%%%%%%%%
%%%%%%%%%%%%%%%%%%%%%%%%%%%%%%%%%%%%%%%%%%%%%%%%%%%%%%%%%%%%%%%%%%%%%%%%%%%%%%%
%%%%%%%%%%%%%%%%%%%%%%%%%%%%%%%%%%%%%%%%%%%%%%%%%%%%%%%%%%%%%%%%%%%%%%%%%%%%%%%
\begin{frame}[fragile]{¿Cómo trabaja?}
  \begin{itemize}
  \item Escribe tu documento en  \texttt{texto plano} con \cmd{comandos} que describen su estructura y significado.
  \item El programa \texttt{latex} procesa su texto y  comandos para
    producir un documento de alta calidad tipográfica.
  \end{itemize}
  \vskip 2ex
  \begin{center}
    \begin{minted}[frame=single]{latex}
La lluvia en Espa\~na cae \emph{principalmente} 
en la llanura.
    \end{minted}
    \vskip 2ex
    \tikz\node[single arrow,fill=gray,font=\ttfamily\bfseries,%
    rotate=270,xshift=-1em]{latex};
    \vskip 2ex
    \fbox{La lluvia en España cae \emph{principalmente} sobre la llanura.}
  \end{center}
\end{frame}

%%%%%%%%%%%%%%%%%%%%%%%%%%%%%%%%%%%%%%%%%%%%%%%%%%%%%%%%%%%%%%%%%%%%%%%%%%%%%%%
%%%%%%%%%%%%%%%%%%%%%%%%%%%%%%%%%%%%%%%%%%%%%%%%%%%%%%%%%%%%%%%%%%%%%%%%%%%%%%%
%%%%%%%%%%%%%%%%%%%%%%%%%%%%%%%%%%%%%%%%%%%%%%%%%%%%%%%%%%%%%%%%%%%%%%%%%%%%%%%
\begin{frame}[fragile]{Más ejemplos de comandos y sus salidas\ldots}
  \begin{exampletwoup}
\begin{itemize}
\item T\'e
\item Leche
\item Galletas
\end{itemize}
  \end{exampletwoup}
  \vskip 2ex
  \begin{exampletwoup}
\begin{figure}
  \includegraphics{es/chick}
\end{figure}
  \end{exampletwoup}
  \vskip 2ex
  \begin{exampletwoup}
\begin{equation}
  \alpha + \beta + 1
\end{equation}
  \end{exampletwoup}
  
  \tiny{Imagen de \url{http://www.andy-roberts.net/writing/latex/importing_images}}
\end{frame}

%%%%%%%%%%%%%%%%%%%%%%%%%%%%%%%%%%%%%%%%%%%%%%%%%%%%%%%%%%%%%%%%%%%%%%%%%%%%%%% 
%%%%%%%%%%%%%%%%%%%%%%%%%%%%%%%%%%%%%%%%%%%%%%%%%%%%%%%%%%%%%%%%%%%%%%%%%%%%%%% 
%%%%%%%%%%%%%%%%%%%%%%%%%%%%%%%%%%%%%%%%%%%%%%%%%%%%%%%%%%%%%%%%%%%%%%%%%%%%%%%
\begin{frame}[fragile]{Cambio de concepto en la redacción}
  
  \begin{itemize}
  \item Utilizar comandos para describir ``Qué es'', y no ``Cómo se ve''.
  \item Concentrarse en su contenido.
  \item Deje a \LaTeX{} hacer su trabajo.
  \end{itemize}
\end{frame}

%%%%%%%%%%%%%%%%%%%%%%%%%%%%%%%%%%%%%%%%%%%%%%%%%%%%%%%%%%%%%%%%%%%%%%%%%%%%%%%
%%%%%%%%%%%%%%%%%%%%%%%%%%%%%%%%%%%%%%%%%%%%%%%%%%%%%%%%%%%%%%%%%%%%%%%%%%%%%%%
%%%%%%%%%%%%%%%%%%%%%%%%%%%%%%%%%%%%%%%%%%%%%%%%%%%%%%%%%%%%%%%%%%%%%%%%%%%%%%%
\section{Conceptos Básicos}

%%%%%%%%%%%%%%%%%%%%%%%%%%%%%%%%%%%%%%%%%%%%%%%%%%%%%%%%%%%%%%%%%%%%%%%%%%%%%%%
%%%%%%%%%%%%%%%%%%%%%%%%%%%%%%%%%%%%%%%%%%%%%%%%%%%%%%%%%%%%%%%%%%%%%%%%%%%%%%%
%%%%%%%%%%%%%%%%%%%%%%%%%%%%%%%%%%%%%%%%%%%%%%%%%%%%%%%%%%%%%%%%%%%%%%%%%%%%%%%
\subsection{Comenzamos}
\begin{frame}[fragile]{\insertsubsection}
  \begin{itemize}
  \item Un documento \LaTeX{} simple:
    \inputminted[frame=single]{latex}{es/basics.tex}
  \item Los comandos comienzan con una \emph{barra invertida} \keystrokebftt{\bs}.
  \item Todo documento comienza con un comando \cmdbs{documentclass}.
  \item El \emph{argumento} en llaves \keystrokebftt{\{}
    \keystrokebftt{\}} le dice a \LaTeX{} que tipo de documento estamos
    creando: en este ejemplo, \bftt{article}.
  \item Un signo de porcentaje \keystrokebftt{\%} comienza un
    \emph{comentario} --- \LaTeX{} ignorará el resto de la línea.
  \end{itemize}
\end{frame}

%%%%%%%%%%%%%%%%%%%%%%%%%%%%%%%%%%%%%%%%%%%%%%%%%%%%%%%%%%%%%%%%%%%%%%%%%%%%%%% 
%%%%%%%%%%%%%%%%%%%%%%%%%%%%%%%%%%%%%%%%%%%%%%%%%%%%%%%%%%%%%%%%%%%%%%%%%%%%%%%
%%%%%%%%%%%%%%%%%%%%%%%%%%%%%%%%%%%%%%%%%%%%%%%%%%%%%%%%%%%%%%%%%%%%%%%%%%%%%%%
\begin{frame}[fragile]{\insertsubsection{} con \wllogo}
  \begin{itemize}
  \item Overleaf es un sitio web para escribir documentos en \LaTeX.
  \item Este ``compila'' su texto \LaTeX{} automáticamente para
    mostrarle el resultado.
    \vskip 2em
    \begin{center}
      \fbox{\href{\wlnewdoc{es/basics.tex}}{%
          Click aquí para abrir el documento de ejemplo en \wllogo{}}}
      \\[1ex]\scriptsize{}
      Para un mejor resultado, use
      \href{http://www.google.com/chrome}{Google Chrome} o una versión
      actualizada de \href{http://www.mozilla.org/en-US/firefox/new/}{FireFox}.
    \end{center}
    \vskip 2ex
  \item A medida que avancemos a través de las siguientes
    diapositivas, prueba los ejemplos escribiéndolos sobre la
    plataforma Overleaf.
  \item \textbf{No, en serio, debería probarlos a medida que avancemos!}
  \end{itemize}
\end{frame}

%%%%%%%%%%%%%%%%%%%%%%%%%%%%%%%%%%%%%%%%%%%%%%%%%%%%%%%%%%%%%%%%%%%%%%%%%%%%%%% 
%%%%%%%%%%%%%%%%%%%%%%%%%%%%%%%%%%%%%%%%%%%%%%%%%%%%%%%%%%%%%%%%%%%%%%%%%%%%%%%
%%%%%%%%%%%%%%%%%%%%%%%%%%%%%%%%%%%%%%%%%%%%%%%%%%%%%%%%%%%%%%%%%%%%%%%%%%%%%%%
\subsection{Tipográfica de Texto}
\begin{frame}[fragile]{\insertsubsection{}}
  \small
  \begin{itemize}
  \item Escriba su texto entre\cmdbegin{document} y \cmdend{document}.
  \item En su mayoría, puede escribir texto normalmente.
    \begin{exampletwouptiny}
Las palabras se separan por uno 
o m\'as espacios.

Los p\'arrafos se separan por 
uno o m\'as lineas en blanco.
    \end{exampletwouptiny}
  \item Los espacios de más en el archivo fuentes son eliminados en la salida. 
    \begin{exampletwouptiny}
La   lluvia    en Espa\~na
cae principalmente   sobre 
la llanura.
    \end{exampletwouptiny}
  \end{itemize}
\end{frame}

%%%%%%%%%%%%%%%%%%%%%%%%%%%%%%%%%%%%%%%%%%%%%%%%%%%%%%%%%%%%%%%%%%%%%%%%%%%%%%% 
%%%%%%%%%%%%%%%%%%%%%%%%%%%%%%%%%%%%%%%%%%%%%%%%%%%%%%%%%%%%%%%%%%%%%%%%%%%%%%% 
%%%%%%%%%%%%%%%%%%%%%%%%%%%%%%%%%%%%%%%%%%%%%%%%%%%%%%%%%%%%%%%%%%%%%%%%%%%%%%%
\begin{frame}[fragile]{\insertsubsection{}: Aclaraciones}
  \small
  \begin{itemize}
  \item Las comillas son un poco complicadas: use el acento invertido
    \keystroke{\`{}} sobre el lado izquierdo y el apóstrofe
    \keystroke{\'{}} sobre el lado derecho.
    \begin{exampletwouptiny}
 Comillas simple: `texto'.

 Comillas dobles: ``texto''.
    \end{exampletwouptiny}
    
  \item Algunos caracteres comunes tienen significados especiales en \LaTeX:\\[1ex]
    \begin{tabular}{cl}
      \keystrokebftt{\%} & Signo de porcentaje \\
      \keystrokebftt{\#} & Signo numeral \\
      \keystrokebftt{\&} & Ampersand                 \\
      \keystrokebftt{\$} & Signo pesos               \\
    \end{tabular}
  \item Si son usados, tendremos errores en la compilación. Si quieres
    que alguno de estos caracteres aparezcan en la salida, se tiene que
    preceder con una barra invertida al caracter.
    \begin{exampletwoup}
\$\%\&\#!
    \end{exampletwoup}
  \end{itemize}
\end{frame}

%%%%%%%%%%%%%%%%%%%%%%%%%%%%%%%%%%%%%%%%%%%%%%%%%%%%%%%%%%%%%%%%%%%%%%%%%%%%%%% 
%%%%%%%%%%%%%%%%%%%%%%%%%%%%%%%%%%%%%%%%%%%%%%%%%%%%%%%%%%%%%%%%%%%%%%%%%%%%%%% 
%%%%%%%%%%%%%%%%%%%%%%%%%%%%%%%%%%%%%%%%%%%%%%%%%%%%%%%%%%%%%%%%%%%%%%%%%%%%%%% 
\begin{frame}[fragile]{Errores de manejo}
  \begin{itemize}
  \item \LaTeX{} puede confundirse cuando está intentando compilar su
    documento. Si esto sucede, se detendrá la compilación por un
    error, y en este caso deberá corregir antes de producir cualquier
    archivo de salida. 
  \item Por ejemplo, si escribe mal \cmdbs{emph} como \cmdbs{meph},
    \LaTeX{} se detendrá con un mensaje de error ``undefined control
    sequence'', ya que ``meph'' no es un comando reconocido.
  \end{itemize}
  \begin{block}{Indicaciones sobre Errores}
    \begin{enumerate}
    \item No se asuste! Los errores suceden.
    \item Corregirlos a medida que se vayan presentando --- si lo que
      acabas de escribir causa un error, puedes comenzar a depurar por ahí.
    \item Si hay múltiples errores, comienza por el primero de ellos
      --- La causa puede incluso estar por arriba de este.
    \end{enumerate}
  \end{block}
\end{frame}

%%%%%%%%%%%%%%%%%%%%%%%%%%%%%%%%%%%%%%%%%%%%%%%%%%%%%%%%%%%%%%%%%%%%%%%%%%%%%%%
%%%%%%%%%%%%%%%%%%%%%%%%%%%%%%%%%%%%%%%%%%%%%%%%%%%%%%%%%%%%%%%%%%%%%%%%%%%%%%%
%%%%%%%%%%%%%%%%%%%%%%%%%%%%%%%%%%%%%%%%%%%%%%%%%%%%%%%%%%%%%%%%%%%%%%%%%%%%%%%
\begin{frame}[fragile]{Ejercicio de Tipografía 1}

  \begin{block}{Escriba esto en \LaTeX:
      \footnote{\url{http://en.wikipedia.org/wiki/Economy_of_the_United_States}}}
    In March 2006, Congress raised that ceiling an additional \$0.79 trillion to
    \$8.97 trillion, which is approximately 68\% of GDP. As of October 4, 2008, the
    ``Emergency Economic Stabilization Act of 2008'' raised the current debt ceiling
    to \$11.3 trillion.
  \end{block}
  \vskip 2ex
  \begin{center}
    \fbox{\href{\wlnewdoc{es/basics-exercise-1.tex}}{%
        Click para abrir este ejercicio en \wllogo{}}}
  \end{center}
  
  \begin{itemize}
  \item Consejo: Tenga cuidado con los caracteres con significados especiales!
  \item Una vez que lo haya probado,
    \fbox{\href{\wlnewdoc{es/basics-exercise-1-solution.tex}}{%
        click aquí para ver la solución}}.
  \end{itemize}
\end{frame}

%%%%%%%%%%%%%%%%%%%%%%%%%%%%%%%%%%%%%%%%%%%%%%%%%%%%%%%%%%%%%%%%%%%%%%%%%%%%%%%
%%%%%%%%%%%%%%%%%%%%%%%%%%%%%%%%%%%%%%%%%%%%%%%%%%%%%%%%%%%%%%%%%%%%%%%%%%%%%%%
%%%%%%%%%%%%%%%%%%%%%%%%%%%%%%%%%%%%%%%%%%%%%%%%%%%%%%%%%%%%%%%%%%%%%%%%%%%%%%%
\subsection{Tipografía Matemática}
\begin{frame}[fragile]{\insertsubsection{}: Signo pesos}
  \begin{itemize}
  \item ¿Por qué son especiales los signos pesos \keystrokebftt{\$}? Los
    usamos para marcar contenido  matemático en el texto.\\[1ex]
    \begin{exampletwouptiny}
% no tan bueno:
Sean a y b distintos n\'umeros 
enteros positivos, y digamos 
que c = a - b + 1.
      
% mucho mejor:
Sean $a$ y $b$ distintos n\'umeros
enteros positivos, y digamos 
que  $c = a - b + 1$.
    \end{exampletwouptiny}
  \item Utilice siempre los signos de pesos en pares --- uno para
    comenzar el contenido matemático, y uno para terminarlo.
  \item \LaTeX{} maneja el espacio automáticamente; por lo que
    ignorará los que hayamos puesto.
    \begin{exampletwouptiny}
Sea $y=mx+b$ \ldots
      
Sea $y = m x + b$ \ldots
    \end{exampletwouptiny}
  \end{itemize}
\end{frame}

%%%%%%%%%%%%%%%%%%%%%%%%%%%%%%%%%%%%%%%%%%%%%%%%%%%%%%%%%%%%%%%%%%%%%%%%%%%%%%%
%%%%%%%%%%%%%%%%%%%%%%%%%%%%%%%%%%%%%%%%%%%%%%%%%%%%%%%%%%%%%%%%%%%%%%%%%%%%%%%
%%%%%%%%%%%%%%%%%%%%%%%%%%%%%%%%%%%%%%%%%%%%%%%%%%%%%%%%%%%%%%%%%%%%%%%%%%%%%%%
\begin{frame}[fragile]{\insertsubsection{}: Notación}
  \begin{itemize}
  \item Use el signo \keystrokebftt{\^} para indicar superíndices y el
    guión bajo \keystrokebftt{\_} para marcar subíndices.

    \begin{exampletwouptiny}
$y = c_2 x^2 + c_1 x + c_0$
    \end{exampletwouptiny}
    \vskip 2ex
    
  \item Utilice las llaves \keystrokebftt{\{} \keystrokebftt{\}} para
    agrupar superíndices y subíndices.
    \begin{exampletwouptiny}
$F_n = F_n-1 + F_n-2$     % oops!
      
$F_n = F_{n-1} + F_{n-2}$ % ok!
    \end{exampletwouptiny}
    \vskip 2ex
    
  \item Hay comandos para letras Griegas y notación común.
    \begin{exampletwouptiny}
$\mu = A e^{Q/RT}$
      
$\Omega = \sum_{k=1}^{n} \omega_k$
    \end{exampletwouptiny}
  \end{itemize}
\end{frame}

%%%%%%%%%%%%%%%%%%%%%%%%%%%%%%%%%%%%%%%%%%%%%%%%%%%%%%%%%%%%%%%%%%%%%%%%%%%%%%%
%%%%%%%%%%%%%%%%%%%%%%%%%%%%%%%%%%%%%%%%%%%%%%%%%%%%%%%%%%%%%%%%%%%%%%%%%%%%%%%
%%%%%%%%%%%%%%%%%%%%%%%%%%%%%%%%%%%%%%%%%%%%%%%%%%%%%%%%%%%%%%%%%%%%%%%%%%%%%%%
\begin{frame}[fragile]{\insertsubsection{}: Ecuaciones}
  \begin{itemize}
  \item Si la ecuación es grande y compleja, se lo puede
    \emph{visualizar} en varias lineas usando \cmdbegin{equation} y
    \cmdend{equation}.\\[2ex]
    \begin{exampletwouptiny}
Las ra\'ices de una ecuaci\'on
cuadr\'atica est\'an dadas por 
\begin{equation}
x = \frac{-b \pm \sqrt{b^2 - 4ac}}
    {2a}
\end{equation}
donde $a$, $b$ and $c$ son \ldots
    \end{exampletwouptiny}
    \vskip 1em
    {\scriptsize Cuidado: Las mayorías de las veces \LaTeX{} ignora los
      espacios en modo matemático, pero no puede manejar líneas en
      blanco en las ecuaciones --- no ponga líneas en blanco en sus
      textos matemáticos.}
  \end{itemize}
\end{frame}

%%%%%%%%%%%%%%%%%%%%%%%%%%%%%%%%%%%%%%%%%%%%%%%%%%%%%%%%%%%%%%%%%%%%%%%%%%%%%%% 
%%%%%%%%%%%%%%%%%%%%%%%%%%%%%%%%%%%%%%%%%%%%%%%%%%%%%%%%%%%%%%%%%%%%%%%%%%%%%%%
%%%%%%%%%%%%%%%%%%%%%%%%%%%%%%%%%%%%%%%%%%%%%%%%%%%%%%%%%%%%%%%%%%%%%%%%%%%%%%%
\begin{frame}[fragile]{Intermedio: Entornos}
  \begin{itemize}
  \item \bftt{equation} es un \emph{entorno} --- un contexto.
  \item Un comando puede producir diferentes salidas en diferentes contextos.

    \begin{exampletwouptiny}
Podemos escribir
$ \Omega = \sum_{k=1}^{n} \omega_k $
en nuestro texto, o podemos escribir
\begin{equation}
  \Omega = \sum_{k=1}^{n} \omega_k
\end{equation}
para mostrarlo en un entorno diferente.
    \end{exampletwouptiny}
    \vskip 2ex
  \item Note como el $\Sigma$ es más grande en el entorno
    \bftt{equation}, y como el subíndice y superíndice cambian de
    posición, a pesar de que utilizamos los mismos comandos.
    \vskip 1em
    {\scriptsize Incluso, podríamos haber escrito \bftt{\$...\$} como
      \cmdbegin{math}\bftt{...}\cmdend{math}.}
  \end{itemize}
\end{frame}

%%%%%%%%%%%%%%%%%%%%%%%%%%%%%%%%%%%%%%%%%%%%%%%%%%%%%%%%%%%%%%%%%%%%%%%%%%%%%%%
%%%%%%%%%%%%%%%%%%%%%%%%%%%%%%%%%%%%%%%%%%%%%%%%%%%%%%%%%%%%%%%%%%%%%%%%%%%%%%%
%%%%%%%%%%%%%%%%%%%%%%%%%%%%%%%%%%%%%%%%%%%%%%%%%%%%%%%%%%%%%%%%%%%%%%%%%%%%%%%
\begin{frame}[fragile]{Intermedio: Entornos}
  \begin{itemize}
  \item Los comandos \cmdbs{begin} y \cmdbs{end} son usados para crear
    muchos entornos diferentes.
    \vskip 2ex
    
  \item Los entornos \bftt{itemize} y \bftt{enumerate} generan listas.
    \begin{exampletwouptiny}
\begin{itemize} % por vi\~netas
\item Galletas
\item T\'e
\end{itemize}
      
\begin{enumerate} % por n\'umeros
\item Galletas
\item T\'e
\end{enumerate}
   \end{exampletwouptiny}
  \end{itemize}
\end{frame}

%%%%%%%%%%%%%%%%%%%%%%%%%%%%%%%%%%%%%%%%%%%%%%%%%%%%%%%%%%%%%%%%%%%%%%%%%%%%%%%
%%%%%%%%%%%%%%%%%%%%%%%%%%%%%%%%%%%%%%%%%%%%%%%%%%%%%%%%%%%%%%%%%%%%%%%%%%%%%%%
%%%%%%%%%%%%%%%%%%%%%%%%%%%%%%%%%%%%%%%%%%%%%%%%%%%%%%%%%%%%%%%%%%%%%%%%%%%%%%% 
\begin{frame}[fragile]{Intermedio: Paquetes}
  
  \begin{itemize}
  \item Todos los comandos y entornos que hemos utilizado hasta el
    momento se encuentran integrados en \LaTeX.
    
  \item Los \emph{paquetes} son librerías de comandos y entornos
    adicionales. Hay miles de paquetes de libre acceso.
    
  \item Tenemos que cargar cada uno de los paquetes que deseamos usar
    con el comando \cmdbs{usepackage}  en el \emph{preámbulo}.
    
  \item Ejemplo: \bftt{amsmath} desde la American Mathematical Society.
    \begin{minted}[fontsize=\small,frame=single]{latex}
\documentclass{article}
\usepackage{amsmath} % pre\'ambulo
\begin{document}
% ahora podemos usar los comandos desde el 
% paquete amsmath...
\end{document}
  \end{minted}
\end{itemize}
\end{frame}

%%%%%%%%%%%%%%%%%%%%%%%%%%%%%%%%%%%%%%%%%%%%%%%%%%%%%%%%%%%%%%%%%%%%%%%%%%%%%%%
%%%%%%%%%%%%%%%%%%%%%%%%%%%%%%%%%%%%%%%%%%%%%%%%%%%%%%%%%%%%%%%%%%%%%%%%%%%%%%%
%%%%%%%%%%%%%%%%%%%%%%%%%%%%%%%%%%%%%%%%%%%%%%%%%%%%%%%%%%%%%%%%%%%%%%%%%%%%%%%
\begin{frame}[fragile]{\insertsubsection{}: Ejemplos con \bftt{amsmath}}
  \begin{itemize}
  \item Utilice \bftt{equation*} (``ecuación-asterisco'') para
    ecuaciones no-numeradas.
    \begin{exampletwouptiny}
\begin{equation*}
 \Omega = \sum_{k=1}^{n} \omega_k
\end{equation*}
    \end{exampletwouptiny}
  \item \LaTeX{} trata las letras adyacentes como variables
    multiplicadas entre sí, lo cual no siempre es lo que se
    quiere. \bftt{amsmath} define comandos para muchos operadores
    matemáticos comunes.
    \begin{exampletwouptiny}
\begin{equation*} % bad!
  min_{x,y} (1-x)^2 + 100(y-x^2)^2
\end{equation*}
\begin{equation*} % good!
 \min_{x,y}{(1-x)^2 + 100(y-x^2)^2}
\end{equation*}
    \end{exampletwouptiny}
  \item Puede utilizar \cmdbs{operatorname} para otros.
    \begin{exampletwouptiny}
\begin{equation*}
 \beta_i =
 \frac{\operatorname{Cov}(R_i, R_m)}
      {\operatorname{Var}(R_m)}
\end{equation*}
    \end{exampletwouptiny}
  \end{itemize}
\end{frame}

%%%%%%%%%%%%%%%%%%%%%%%%%%%%%%%%%%%%%%%%%%%%%%%%%%%%%%%%%%%%%%%%%%%%%%%%%%%%%%%
%%%%%%%%%%%%%%%%%%%%%%%%%%%%%%%%%%%%%%%%%%%%%%%%%%%%%%%%%%%%%%%%%%%%%%%%%%%%%%%
%%%%%%%%%%%%%%%%%%%%%%%%%%%%%%%%%%%%%%%%%%%%%%%%%%%%%%%%%%%%%%%%%%%%%%%%%%%%%%%
\begin{frame}[fragile]{\insertsubsection{}: Ejemplos con \bftt{amsmath}}
  \begin{itemize}{\small
    \item Alinear una secuencia de ecuaciones al signo igual
      \begin{align*}
        (x+1)^3 &= (x+1)(x+1)(x+1) \\
        &= (x+1)(x^2 + 2x + 1) \\
        &= x^3 + 3x^2 + 3x + 1
      \end{align*}
      con el entorno \bftt{align*}.
      
      % for whatever reason, this doesn't play well with the twoup environment
      \begin{minted}[fontsize=\small,frame=single]{latex}
\begin{align*}
  (x+1)^3 &= (x+1)(x+1)(x+1) \\
  &= (x+1)(x^2 + 2x + 1) \\
  &= x^3 + 3x^2 + 3x + 1
\end{align*}
      \end{minted}
    \item El ampersand \keystrokebftt{\&} separa la columna izquierda
      (antes del $=$) de la columna derecha (después del $=$).
    \item Una doble barra invertida
      \keystrokebftt{\bs}\keystrokebftt{\bs} da comienzo a una nueva línea.
    }\end{itemize}
\end{frame}


%%%%%%%%%%%%%%%%%%%%%%%%%%%%%%%%%%%%%%%%%%%%%%%%%%%%%%%%%%%%%%%%%%%%%%%%%%%%%%%
%%%%%%%%%%%%%%%%%%%%%%%%%%%%%%%%%%%%%%%%%%%%%%%%%%%%%%%%%%%%%%%%%%%%%%%%%%%%%%%
%%%%%%%%%%%%%%%%%%%%%%%%%%%%%%%%%%%%%%%%%%%%%%%%%%%%%%%%%%%%%%%%%%%%%%%%%%%%%%%
\begin{frame}[fragile]{Ejercicio de Tipografía 2}
  
  \begin{block}{Escriba esto en  \LaTeX:}
    Sean $X_1, X_2, \ldots, X_n$ una secuencia de variables aleatorias
    independientes e id\'enticamente distribuidas con  $\operatorname{E}[X_i] = \mu$ y
    $\operatorname{Var}[X_i] = \sigma^2 < \infty$, y sea
    \begin{equation*}
      S_n = \frac{1}{n}\sum_{i}^{n} X_i
    \end{equation*}
    indica su media. Entonces, cuando $n$ tienda al infinito, las
    variables aleatorias $\sqrt{n}(S_n - \mu)$ convergen en la
    distribución a una normal $N(0, \sigma^2)$.
  \end{block}
  \vskip 2ex
  \begin{center}
    \fbox{\href{\wlnewdoc{es/basics-exercise-2.tex}}{%
        Click to open this exercise in \wllogo{}}}
  \end{center}
  \begin{itemize}
  \item Consejo: el comando para $\infty$ es \cmdbs{infty}.
  \item Una vez que lo haya probado,
    \fbox{\href{\wlnewdoc{es/basics-exercise-2-solution.tex}}{%
        click aquí para ver la solución}}.
  \end{itemize}
\end{frame}

%%%%%%%%%%%%%%%%%%%%%%%%%%%%%%%%%%%%%%%%%%%%%%%%%%%%%%%%%%%%%%%%%%%%%%%%%%%%%%% 
%%%%%%%%%%%%%%%%%%%%%%%%%%%%%%%%%%%%%%%%%%%%%%%%%%%%%%%%%%%%%%%%%%%%%%%%%%%%%%%
%%%%%%%%%%%%%%%%%%%%%%%%%%%%%%%%%%%%%%%%%%%%%%%%%%%%%%%%%%%%%%%%%%%%%%%%%%%%%%%
\begin{frame}{Final de la Parte 1}
  \begin{itemize}
  \item Felicitaciones! Ya que has aprendido cómo \ldots
    \begin{itemize}
    \item Componer texto en \LaTeX.
    \item Utilizar diferentes comandos.
    \item Controlar los errores que puedan surgir.
    \item Componer contenido matemático de alta calidad.
    \item Utilizar varios diferentes entornos.
    \item Cargar paquetes.
    \end{itemize}
  \item Eso es increíble!
  \item En la Parte 2, veremos como usar \LaTeX{} para escribir
    documentos estructurados con secciones, referencias cruzadas,
    figuras, tablas y bibliografías. ¡Hasta entonces!
  \end{itemize}
\end{frame}

\end{document}